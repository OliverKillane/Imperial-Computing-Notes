\chapter{Estimators And Confidence Intervals}

\section{Efficient Consistent Estimator}
We can quantify how \textit{good} estimators are. For example with the \keyword{Estimator Bias} (difference
between the expected using the estimator and the parameter $bias(T) = E[T|\theta] - \theta$). We also wanto to
quantify the \keyword{Efficiency of Estimators}.
\begin{definitionbox}{Estimator Efficiency}
	Given two unbiased estimators $\hat{\Theta}(\underline{X})$ and $\tilde{\Theta}(\underline{X})$ where $\underline{X} = (X_1, \dots, X_n)$ (a sample containing $n$ observations $X\dots$).
	\\
	\\ We can compare the mean, variances etc to determine which estimator is more efficient (typically lower variance)
	\\
	\\ $\hat{\Theta}$ is more efficient than $\tilde{\Theta}$ if:
	\[\forall \theta Var_{\hat{\Theta}}(\hat{\Theta}| \theta) \leq Var_{\tilde{\Theta} | \theta}(\tilde{\Theta} | \theta) \
		\ \text{ or }
		\ \
		\exists \theta Var_{\hat{\Theta}}(\hat{\Theta}| \theta) < Var_{\tilde{\Theta} | \theta}(\tilde{\Theta} | \theta)
	\]
	More efficient means less variance in estimates.
	\\
	\\ IF an estimator is more efficient than any other possible estimator, it is called \keyword{efficient}.
\end{definitionbox}
\begin{examplebox}{Bias and Efficiency}
	Given a population with mean $\mu$ and variance $\sigma^2$. We have a sample:
	\[\underline{X} = (X_1, \dots, X_n)\]
	We consider two estimators:
	\begin{enumerate}
		\item $\hat{M} = \overline{X}$ (the sample mean)
		\item $\tilde{M} = X_1$ (the first observation in the sample)
    \end{enumerate}
	We can compute the bias as for both:
	\begin{enumerate}
		\item The expected value of the sample mean is the population mean $\mu$, hence $\hat{M}$ is unbiased.
		\item The expected value of any observation is $\mu$, so the first observation in the sample is also ubiased.
	\end{enumerate}
	Next we can consider the variance.
	\\
	\\ For a single sample we know the variance will be $\sigma^2$, hence:
	\[Var_{\tilde{M}}(\tilde{M}|\mu \text{ and } \sigma^2) = Var(X_1) = \sigma^2\]
	However for the sample mean, we know can use the \keyword{Central Limit Theorem} to determine that the variance of the mean of a sample will be divided by the sample size.
	\[Var_{\hat{M}}(\hat{M}|\mu \text{ and } \sigma^2) = Var(\overline{X}) = \cfrac{\sigma^2}{n}\]
	Hence for all values of $n$, the variance of $\hat{M} \leq \tilde{M}$ (at $n=1$ they are equal), so $\hat{M}$ is the more efficient estimator.
\end{examplebox}
\begin{definitionbox}{Estimator Consistency}
	A consistent estimator improves as the sample size grows. Formally:
	\[\forall \epsilon > 0 \ P(|\hat{\Theta} - \theta|) \to 0 \ \text{ as } \ n \to \infty\]
	If $\hat{\Theta}$ is unbiased, then:
	\[\lim_{n \to \infty} Var(\hat{\Theta}) = 0 \Rightarrow \hat{\Theta} \ \text{ is consistent}\]
	Note: $\overline{X}$ (sample mean) is a consistent estimator for any population.
\end{definitionbox}
\section{Confidence Intervals}
\centerimage{width=0.75\textwidth}{estimators_and_confidence_intervals/images/estimate.drawio.png}
In order to quantify our degree of uncertainty in an estimate $\hat{\theta}$, when the true value $\theta$ is unknown, we use use our estimate as the true value, to compute the distribution $P_{T|\hat{\theta}}$ (the approximate sampling distribution).
\subsection{Known Variance}
\subsubsection{Confidence Interval}
If we know the true variance of the population, then the sample mean would be distributed as:
\[\overline{X} \thicksim N\left(\overline{x}, \cfrac{\sigma^2}{n} \right)\]
If $\mu$ (population mean) $= \overline{x}$, then we can say that (using the standard normal distribution) there is a $95\%$ probability the observed statistic $\overline{X}$ is in the range:
\[\left[\overline{x} - 1.96\cfrac{\sigma}{n}, \overline{x} + 1.96\cfrac{\sigma}{n} \right]\]
(Double ended, $95\%$ confidence interval for $\mu$)
\centerimage{width=0.6\textwidth}{estimators_and_confidence_intervals/images/confidence_interval.drawio.png}
\subsubsection{With the Standard Normal Distribution}
We can define any normal distribution in terms of the standard normal distribution.
\[X \thicksim N(\mu, \sigma^2) \Leftrightarrow Y = \cfrac{X - \mu}{\sigma} \Leftrightarrow Y \thicksim N(0,1)\]
We can then use tables for the standard normal distribution, using $\Phi(z) = P(X \leq z)$ given $Z \in N(0,1)$:
\\
\\ Note if you have sample size as part of the variance, $Y = \cfrac{X - \mu}{\left(\cfrac{\sigma}{\sqrt{n}}\right)}$.
\\
\\ For example in the previous confidence interval, we used the normal distribution to calculate the values.
\centerimage{width=0.6\textwidth}{estimators_and_confidence_intervals/images/confidence_standard_normal.drawio.png}
Given the critical value $z$ for the normal distribution e.g $1.96$ for double-ended $95\%$ confidence interval, we have:
\begin{center}
	\begin{tabular}{r c c}
		Standard Normal     & $X \thicksim N(0,1)$                                            & $[-z, z]$                                                                                         \\
		Normal Distribution & $X \thicksim N(\mu, \sigma^2)$                                  & $\mu - z\sigma, \mu + z\sigma$                                                                    \\
		Sample Mean         & $\overline{X} \thicksim N\left(\mu, \cfrac{\sigma^2}{n}\right)$ & $\left[\mu - z\cfrac{\sigma}{\sqrt{n}}, \mu + z\cfrac{\sigma}{\sqrt{n}}\right]$                   \\
		\\
		Population mean     & $\mu \thicksim N\left(\overline{X}, \cfrac{\sigma^2}{n}\right)$ & $\left[\overline{x} - z\cfrac{\sigma}{\sqrt{n}}, \overline{x} + z\cfrac{\sigma}{\sqrt{n}}\right]$ \\
	\end{tabular}
\end{center}
\begin{examplebox}{Employees Opinions on the Board}
	A corporation surveys employees on wether they think the board is doing a good job.
	\\
	\\ $1000$ employees are randomly selected, and $732$ say the board is doing a good job. Find the $99\%$
	confidence interval for the proportion of the employees that think the board is doing a good job. Assume the variance is $\sigma^2 = 0.25$.
	\\
	\\ First we get the sample mean:
	\[\overline{x} = \cfrac{732}{1000} = 0.732\]
	Next we determine the standard deviation:
	\[\sigma = \sqrt{0.25} = 0.5\]
	We want to get the double-ended $99\%$ interval, so each tail will have size $0.005$. By using the standard normal distribution we have $\Phi(2.576) = 0.995$, so $z = 2.576$.
	\\
	\\ Hence we can calculate the interval as:
	\[\begin{split}
			\mu &= \left[\overline{x} - z\cfrac{\sigma}{\sqrt{n}}, \overline{x} + z\cfrac{\sigma}{\sqrt{n}}\right] \\
			&= \left[0.732 - 2.576\cfrac{0.5}{\sqrt{1000}}, 0.732 + 2.576\cfrac{0.5}{\sqrt{1000}}\right] \\
			&= \left[0.732 - 2.576\cfrac{0.5}{\sqrt{1000}}, 0.732 + 2.576\cfrac{0.5}{\sqrt{1000}}\right] \\
			&\approx 0.732 \pm 0.0407 \\
		\end{split}\]
\end{examplebox}

\subsection{Unknown Variance}
In a problem where we are trying to fit a normal distribution, but both the mean and variance are unknown.
\[\text{Bias Corrected Variance} \ S_{n-1} = \sqrt{\cfrac{\sum_{i=1}^{n}(X_i - \overline{X})^2}{n-1}}\]
We use the bias corrected variance of our sample, and as a result must use a different distribution to the normal distribution.
\begin{center}
	\begin{tabular}{c | c}
		\textbf{Normal Distribution ($\sigma$ known)}                                        & \textbf{Studen't t distribution ($\sigma$ unknown)}                                    \\
		\\
		$\cfrac{\overline{X} - \mu}{\left(\cfrac{\sigma}{\sqrt{n}}\right)} \thicksim N(0,1)$ & $\cfrac{\overline{X} - \mu}{\left(\cfrac{s_{n-1}}{\sqrt{n}}\right)} \thicksim t_{n-1}$ \\
	\end{tabular}
\end{center}
In the student's distribution we set degrees of freedom $\nu = n - 1$.
\\
\\ For a double ended confidence $(100 - \alpha)\%$, we compute $t_{\nu = n - 1, \ 1 - \sfrac{\alpha}{2}}$ to find the critical values (the places where the tails start/ the $\alpha$-quantile of $t_\nu$).
\centerimage{width=0.6\textwidth}{estimators_and_confidence_intervals/images/students_distribution.drawio.png}
\[\left[ \overline{x} - t_{\nu = n - 1, \ 1 - \sfrac{\alpha}{2}} \times \cfrac{s_{n-1}}{\sqrt{n}}, \ \overline{x} + t_{\nu = n - 1, \ 1 - \sfrac{\alpha}{2}} \times \cfrac{s_{n-1}}{\sqrt{n}}\right]\]
When using the tables for $t$ values, we use the size we want (e.g $0.975$ for $95\%$ double-ended confidence interval), and then use the degrees of freedom ($n-1$).
