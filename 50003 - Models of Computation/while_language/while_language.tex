\chapter{While Language}

\section{SimpleExp}
We can define a simple expression language (\textit{SimpleExp}) to work on:
\[E \in SimpleExp \ ::= \ n \ | \ E + E \ | \ E \times E \ | \ \dots\]
We want semantics that are the same as we would expect in typical mathematics notation
\begin{tcbraster}[raster columns=2,raster equal height]
    \begin{definitionbox}{Small-Set/Structural}
        Gives a method for evaluating an expression step-by-step.
        \[E \to E'\]
    \end{definitionbox}
    \begin{definitionbox}{Big-Step/Natural}
        Ignores intermediate steps and gives result immediately.
        \[E \Downarrow n\]
    \end{definitionbox}
\end{tcbraster}
\noindent We need big to define big and small step semantics for SimpleExp to describe this, and have those semantics conform to several properties listed.

\subsection{Big-Step Semantics}
\subsubsection{Rules}
\[\bigstep{B-NUM}{}{n}{n} \qquad \qquad \bigstepdef{B-ADD}{E_1 \Downarrow n_1 \quad E_2 \Downarrow n_2}{E_1 + E_2}{n_3}{n_3 = n_1 + n_2}\]
We can similarly define multiplication, subtraction etc.
\subsubsection{Properties}
\begin{tcbraster}[raster columns=2,raster equal height]
    \begin{definitionbox}{Determinacy}
        \[\forall E, n_1, n_2 . \ [E \Downarrow n_1 \land E \Downarrow n_2 \Rightarrow n_1 = n_2]\]
    Expression evaluation is deterministic (only one result possible).
    \end{definitionbox}
    \begin{definitionbox}{Totality}
        \[\forall E. \ \exists n . \ [E \Downarrow n] \]
        Every expression evaluates to something.    
    \end{definitionbox}
\end{tcbraster}

\begin{examplebox}{Break it!}
    How could we break the totality of SimpleExp?
    \tcblower
    \[\bigstepdef{B-NON-TOTAL}{}{true}{true}{}\]
    We can break \textit{totality} by introducing a rule that can always match its output.
    \\ 
    \\ The B-NON-TOTAL rule can be applied indefinitely (possible evaluation path that never finishes).
\end{examplebox}

\begin{examplebox}{Now it all adds up!}
    Show that $3 + (2 + 1) \Downarrow 6$ using the provided rules.
    \tcblower
    We can hence create the derivation:
    \[\bigstep{B-ADD}{
        \bigstep{B-NUM}{}{3}{3} \ \bigstep{B-ADD}{
            \bigstep{B-NUM}{}{2}{2} \ \bigstep{B-NUM}{}{1}{1}
        }{2 + 1}{3}
    }{3 + (2 + 1)}{6}\] 
\end{examplebox}

\subsection{Small Step Semantics}
Given a relation $\to$ we can define a its transitive closure $\to^*$ such that:
\[E \to^* E' \Leftrightarrow E = E' \lor \exists E_1, E_2, \dots, E_k . \ [E \to E_1 \to E_2 \to \dots \to E_k \to E'] \]

\subsubsection{Rules}
\[\smallstepdef{S-ADD}{}{n_1 + n_2}{n_3}{n_3 = n_1 + n_2}\]
\[\smallstep{S-LEFT}{E_1 \to E_1'}{E_1 + E_2}{E_1' + E_2} \qquad \smallstep{S-RIGHT}{E \to E'}{n + E}{n + E'}\]
Here we define $+$ as a left-associative operator.
\begin{definitionbox}{Normal Form}
    $E$ is in its normal form (irreducable) if there is no $E'$ such that $E \to E'$        
\end{definitionbox}
\noindent In \textit{SimpleExp} the normal form is the natural numbers.

\subsubsection{Properties}
\begin{definitionbox}{Confluence}
    \[\forall E, E_1, E_2 . \ [E \to^* E_1 \land E \to^* E_2 \Rightarrow \exists E' . \ [E_1 \to^* E' \land E_2 \to* E']]\]
    Determinate $\rightarrow$ Confluent
    \\ There are several evaluations paths, but they all get the same end result.
\end{definitionbox}
\begin{tcbraster}[raster columns=2,raster equal height]
    \begin{definitionbox}{Determinacy}
        \[\forall E, E_1, E_2 . \ [E \to E_1 \land E \to E_2 \Rightarrow E_1 = E_2]\]
        There is at most one next possible step/rule to apply.
    \end{definitionbox}
    \begin{definitionbox}{Strong Normalisation}
        There are no infinite sequences of expressions, all sequences are finite.
    \end{definitionbox}

    % note that the slides have to E_k, assumed to be normal form?
    \begin{definitionbox}{Weak Normalisation}
        \[\forall E. \ \ \exists k. \ \exists n. \ [E \to^k n]\]
        There is some finite sequence of expressions (to normalize) for any expression.
    \end{definitionbox}
    \begin{definitionbox}{Unique Normal Form}
        \[\forall E, n_1, n_2. \ [E \to^* n_1 \land E \to n_2 \Rightarrow n_1 = n_2]\]
    \end{definitionbox}
\end{tcbraster}	

\begin{examplebox}{To be determined\dots}
    Add a rule to break determinacy without breaking confluence.
    \tcblower
    \[\smallstep{S-RIGHT-E}{E_2 \to E_2'}{E_1 + E_2}{E_1 + E_2'}\]
    As we can now choose which side to reduce first (S-LEFT or S-RIGHT-E), we have lost determinacy, however we retain confluence.
\end{examplebox}

\section{While}
\subsection{Syntax}
We can define a simple while language (if, else, while loops) to build programs from \& to analyse.
\begin{center}
	\begin{tabular}{r c l}
		$B \in Bool$ & ::= & $true | false | E = E | E < E | B \& B | \neg B \dots$                   \\
		$E \in Exp$  & ::= & $x | n | E + E | E \times E | \dots$                                     \\
		$C \in Com$  & ::= & $x :=E | if \ B \ then \ C \ else \ C | C;C | skip | while \ B \ do \ C$ \\
	\end{tabular}
\end{center}
Where $x \in Var$ ranges over variable identifiers, and $n \in \mathbb{N}$ ranges over natural numbers.

\subsection{States}
\begin{definitionbox}{Partial Function}
	A mapping of every member of its domain, to at most one member of its codomain.
\end{definitionbox}

A \textit{state} is a partial function from variables to numbers (partial function as only defined for some variables). For state $s$, and variable $x$, $s(x)$ is defined, e.g:
\[s = (x \mapsto 2, y \mapsto 200, z \mapsto 20)\]
(In the current state, $x = 2, y = 200, z = 20$).

For example:
\begin{center}
	\begin{tabular}{r c l c}
		$s[x \mapsto 7](u)$ & $=$ & $7$    & if $u = x$ \\
		                    & $=$ & $s(u)$ & otherwise  \\
	\end{tabular}
\end{center}

The \textit{small-step} semantics of \textit{While} are defined using \textit{configurations} of form:
\[\config{E}{s}, \config{B}{s}, \config{C}{s} \]
(Evaluating $E$, $B$, or $C$ with respect to state $s$)
\\
\\ We can create a new state, where variable $x$ equals value $a$, from an existing state $s$:
\[s'(u) \triangleq \alpha(x)=\begin{cases}
		a    & u = x     \\
		s(u) & otherwise \\
	\end{cases}\]
\[s' = s[x \mapsto u] \text{ is equivalent to } dom(s') = dom(s) \land \forall y.[y \neq x \rightarrow s(y) = s'(y) \land s'(x) = a]\]
($s'$ equals $s$ where $x$ maps to $a$)

\subsection{Rules}
\subsubsection{Expressions}
\[\whilestdef{W-EXP.LEFT}{\config{E_1}{s} \to_e \config{E'_1}{s'}}{\config{E_1 + E_2}{s} \to_e \config{E'_1 + E_2}{s'}}{} \qquad \qquad \whilestdef{W-EXP.RIGHT}{\config{E}{s} \to_e \config{E'}{s'}}{\config{n+E}{s} \to_e \config{n + E'}{s'}}{}\]
\[\whilestdef{W-EXP.VAR}{}{\config{x}{s} \to_e \config{n}{s}}{s(x) = n} \qquad \qquad \whilestdef{W-EXP.ADD}{}{\config{n_1 + n_2}{s}}{\config{n_3}{s}}{n_3 = n_1 + n_2}\]
These rules allow for side effects, despite the While language being side effect free in expression evaluation. We show this by changing state $s \to_e s'$.
\\
\\ We can show inductively (from the base cases W-EXP.VAR and W-EXP.ADD) that expression evaluation is side effect free.
\subsubsection{Booleans}
 (Based on expressions, one can create the same for booleans) ($b \in \{true, false\}$)
\[\whilestdef{W-BOOL.AND.LEFT}{\config{B_1}{s} \to_b \config{B'_1}{s'}}{\config{B_1 \& B_2}{s} \to_b \config{B'_1 \& B_2}{s'}}{} \qquad \whilestdef{W-BOOL.AND.RIGHT}{\config{B}{s} \to_b \config{B'}{s'}}{\config{b \& B_2}{s} \to_b \config{b \& B'}{s'}}{}\]
\[\whilestdef{W-BOOL.AND.TRUE}{}{\config{true \& true}{s} \to_b \config{true}{s}}{} \qquad \whilestdef{W-BOOL.AND.FALSE}{}{\config{false \& b}{s} \to_b \config{true}{s}}{}\]
(Notice we do not short circuit, as the right arm may change the state. In a side effect free language, we could.)
\\
\[\whilestdef{W-BOOL.EQUAL.LEFT}{\config{E_1}{s} \to_e \config{E'_1}{s'}}{\config{E_1 = E_2}{s} \to_b \config{E'_1 = E_2}{s'}}{} \qquad \whilestdef{W-BOOL.EQUAL.RIGHT}{\config{E}{s} \to_e \config{E'}{s'}}{\config{n = E}{s} \to_b \config{n = E}{s'}}{}\]
\[\whilestdef{W-BOOL.EQUAL.TRUE}{}{\config{n_1 = n_2}{s} \to_b \config{true}{s}}{n_1 = n_2} \quad \whilestdef{W-BOOL.EQUAL.FALSE}{}{\config{n_1 = n_2}{s} \to_b \config{false}{s}}{n_1 \neq n_2}\]
\[\whilestdef{W-BOOL.LESS.LEFT}{\config{E_1}{s} \to_e \config{E'_1}{s'}}{\config{E_1 < E_2}{s} \to_b \config{E'_1 < E_2}{s'}}{} \qquad \whilestdef{W-BOOL.LESS.RIGHT}{\config{E}{s} \to_e \config{E'}{s'}}{\config{n < E}{s} \to_b \config{n < E}{s'}}{}\]
\[\whilestdef{W-BOOL.LESS.TRUE}{}{\config{n_1 < n_2}{s} \to_b \config{true}{s}}{n_1 < n_2} \qquad \whilestdef{W-BOOL.EQUAL.FALSE}{}{\config{n_1 < n_2}{s} \to_b \config{false}{s}}{n_1 \geq n_2}\]
\[\whilestdef{W-BOOL.NOT}{}{\config{\neg true}{s} \to_b \config{false}{s}}{} \qquad \whilestdef{W-BOOL.NOT}{}{\config{\neg false}{s} \to_b \config{true}{s}}{}\]

\subsubsection{Assignment}
\[\whilestdef{W-ASS.EXP}{\config{E}{s} \to_e \config{E'}{s'}}{
		\config{x := E}{s} \to_c \config{x := E'}{s'}
	}{} \qquad \whilestdef{W-ASS.NUM}{}{
		\config{x:= n}{s} \to_c \config{skip}{s[x \mapsto n]}
	}{}\]
\subsubsection{Sequential Composition}
\[\whilestdef{W-SEQ.LEFT}{\config{C_1}{s} \to_c \config{C_1'}{s'}}{
		\config{C_1;C_2}{s} \to_c \config{C_1';C_2}{s'}
	}{} \qquad \whilestdef{W-SEQ.SKIP}{}{
		\config{skip;C}{s} \to_c \config{C}{s}
	}{}\]
\subsubsection{Conditionals}
\[\whilestdef{W-COND.TRUE}{}{\config{\text{if } true \text{ then } C_1 \text{ else } C_2}{s} \to_c \config{C_1}{s}}{}\]
\[\whilestdef{W-COND.FALSE}{}{\config{\text{if } false \text{ then } C_1 \text{ else } C_2}{s} \to_c \config{C_2}{s}}{}\]
\[\whilestdef{W-COND.BEXP}{
		\config{B}{s} \to_b \config{B'}{s'}
	}{\config{\text{if } B \text{ then } C_1 \text{ else } C_2}{s} \to_c \config{\text{if } B' \text{ then } C_1 \text{ else } C_2}{s'}}{}\]
\subsubsection{While}
\[\whilestdef{W-WHILE}{}{
		\config{\text{while } B \text{ do } C}{s} \to_c \config{\text{if } B \text{ then } (C; \text{while } B \text{ do } C) \text{ else } skip}{s}
	}{}\]
\subsection{Properties}
The execution relation ($\to_c$) is deterministic.
\[\forall C,C_1,C_2 \in Com \forall s,s_1,s_2 .[\config{C}{s} \to_c \config{C_1}{s_1} \land \config{C}{s} \to_c \config{C_2}{s_2} \rightarrow \config{C_1}{s_1} = \config{C_2}{s_2}]\]
Hence the relation is also confluent:
\[\forall C,C_1,C_2 \in Com \forall s,s_1,s_2 .[\config{C}{s} \to_c \config{C_1}{s_1} \land \config{C}{s} \to_c \config{C_2}{s_2} \rightarrow \]
			\[\exists C' \in Com, s' . [\config{C_1}{s_1} \to_c \config{C'}{s'} \land \config{C_2}{s_2}\to_c \config{C'}{s'} ]]\]
Both also hold for $\to_e$ and $\to_b$.

\subsection{Configurations}
\subsubsection{Answer Configuration}
A configuration $\config{skip}{s}$ is an \textit{answer configuration}. As there is no rule to execute skip, it is a normal form.
\[\neg\exists C \in Com, s, s' . [\config{skip}{s} \to_c \config{C}{s'}]\]
For booleans $\config{true}{s}$ and $\config{false}{s}$ are answer configurations, and for expressions $\config{n}{s}$.
\subsubsection{Stuck Configurations}
A configuration that cannot be evaluated to a normal form is called a \textit{suck configuration}.
\[\config{y}{(x \mapsto 3)}\]
Note that a configuration that leads to a \textit{stuck configuration} is not itself stuck.
\[\config{5 < y}{(x \mapsto 2)}\]
(Not stuck, but reduces to a stuck state)
\subsection{Normalising}
The relations $\to_b$ and $\to_e$ are normalising, but $\to_c$ is not as it may not have a normal form.
\[\text{while } true \text{ do } skip\]
\[\config{\text{while } true \text{ do } skip}{s} \to_c^3 \config{\text{while } true \text{ do } skip}{s}\]
($\to_c^3$ means 3 steps, as we have gone through more than one to get the same configuration, it is an infinite loop)
\subsection{Side Effecting Expressions}
If we allow programs such as:
\[\text{do } x := x + 1 \ return \ x\]
\[(\text{do } x := x + 1 \ return \ x) + (\text{do } x := x \times 1 \ return \ x)\]
(value depends on evaluation order)
\subsection{Short Circuit Semantics}
\[\cfrac{B_1 \to_b B'_1}{B_1 \& B_2 \to_b B_1' \& B_2} \qquad \cfrac{}{false \& B \to_b false} \qquad \cfrac{}{true \& B \to_b B}\]
\subsection{Strictness}
An operation is \textit{strict} when arguments must be evaluated before the operation is evaluated. Addition is struct as both expressions must be evaluated (left, then right).
\\
\\ Due to short circuiting, $\&$ is left strict as it is possible for the operation to be evaluated without evaluating the right (\textit{non-strict} in right argument).

\subsection{Complex Programs}
It is now possible to build complex programs to be evaluated with our small step rules.
\[Factorial \triangleq  y:= x; a:= 1;\text{while }0<y \text{ do } (a:= a \times y; y:= y - 1)\]
We can evaluate $Factorial$ with an input $s = [x \mapsto \dots]$ to get answer configuration $[\dots, a \mapsto x!, x \mapsto \dots]$

\begin{examplebox}{Execute!}
    Evaluate $Factorial$ for the following initial configuration:
    \[s = [x \mapsto 3, y \mapsto 17, z \mapsto 42]\]
    \tcblower
    \subsubsection*{Start}
\[\config{y:= x; a:= 1;\text{while }0<y \text{ do } (a:= a \times y; y:= y - 1)}{[x \mapsto 3, y \mapsto 17, z \mapsto 42]}\]
\subsubsection*{Get x variable}
where $C = a:= 1;\text{while }0<y \text{ do } (a:= a \times y; y:= y - 1)$ and $s = (x \mapsto 3, y \mapsto 17, z \mapsto 42)$:
\[\whilest{W-SEQ.LEFT}{
		\whilest{W-ASS.EXP}{
			\whilest{W-EXP.VAR}{}{\config{x}{s} \to_e \config{3}{s}}
		}{\config{y:= x}{s} \to_c \config{y := 3}{s}}
	}{\config{y:= x; C}{s} \to_c \config{y:=3;C}{s}}\]
Result:
\[\config{y:= 3; a:= 1;\text{while }0<y \text{ do } (a:= a \times y; y:= y - 1)}{(x \mapsto 3, y \mapsto 17, z \mapsto 42)}\]
\subsubsection*{Assign to y variable}
where $C = a:= 1;\text{while }0<y \text{ do } (a:= a \times y; y:= y - 1)$ and $s = (x \mapsto 3, y \mapsto 17, z \mapsto 42)$:
\[\whilest{W-SEQ.LEFT}{
		\whilest{W-ASS.NUM}{}{\config{y:=3}{s} \to_c \config{skip}{s[y \mapsto 3]}}
	}{\config{y:= 3; C}{s} \to_c \config{skip;C}{s[y \mapsto 3]}}\]
Result:
\[\config{skip; a:= 1;\text{while }0<y \text{ do } (a:= a \times y; y:= y - 1)}{(x \mapsto 3, y \mapsto 3, z \mapsto 42)}\]
\subsubsection*{Eliminate skip}
where $C = a:= 1;\text{while }0<y \text{ do } (a:= a \times y; y:= y - 1)$ and $s = (x \mapsto 3, y \mapsto 3, z \mapsto 42)$:
\[\whilest{W-SEQ.SKIP}{}{\config{skip;C}{s} \to_c \config{C}{s}}\]
Result:
\[\config{a:= 1;\text{while }0<y \text{ do } (a:= a \times y; y:= y - 1)}{(x \mapsto 3, y \mapsto 3, z \mapsto 42)}\]
\subsubsection*{Assign a}
where $C = \text{while }0<y \text{ do } (a:= a \times y; y:= y - 1)$ and $s = (x \mapsto 3, y \mapsto 3, z \mapsto 42)$:
\[\whilest{W-SEQ.LEFT}{
		\whilest{W-ASS.NUM}{}{\config{a := 1}{s} \to_c \config{skip}{s[a \mapsto 1]}}
	}{\config{a:=1;C}{s} \to_c \config{skip;C}{s[a \mapsto 1]}}\]
Result:
\[\config{skip;\text{while }0<y \text{ do } (a:= a \times y; y:= y - 1)}{(x \mapsto 3, y \mapsto 3, z \mapsto 42, a \mapsto 1)}\]
\subsubsection*{Eliminate skip}
where $C = \text{while }0<y \text{ do } (a:= a \times y; y:= y - 1)$ and $s = (x \mapsto 3, y \mapsto 3, z \mapsto 42, a \mapsto 1)$
\[\whilest{W-SEQ.SKIP}{}{\config{skip;C}{s} \to_c \config{C}{s}}\]
Result:
\[\config{\text{while }0<y \text{ do } (a:= a \times y; y:= y - 1)}{(x \mapsto 3, y \mapsto 3, z \mapsto 42, a \mapsto 1)}\]
\subsubsection*{Expand while}
where $C = (a:= a \times y; y:= y - 1)$, $B = 0<y$ and $s = (x \mapsto 3, y \mapsto 3, z \mapsto 42, a \mapsto 1)$:
\[\whilest{W-WHILE}{}{\config{\text{while } B \text{ do } C}{s} \to_c \config{\text{if } B \text{ then } (C; \text{while } B \text{ do } C) \text{ else } skip}{s}}\]
Result:
\[\config{\text{if } 0<y \text{ then } (a:= a \times y; y:= y - 1; \text{while } 0<y \text{ do } a:= a \times y; y:= y - 1) \text{ else } skip}{(x \mapsto 3, y \mapsto 3, z \mapsto 42, a \mapsto 1)}\]
\subsubsection*{Get y variable}
where $C = (a:= a \times y; y:= y - 1)$ and $s = (x \mapsto 3, y \mapsto 3, z \mapsto 42, a \mapsto 1)$:
\[\whilest{W-COND.BEXP}{
		\whilest{W-BOOL.LESS.RIGHT}{
			\whilest{W-EXP.VAR}{}{\config{y}{s} \to \config{3}{s}}
		}{\config{0<y}{s} \to_b \config{0<3}{s}}
	}{\config{\text{if } 0<y \text{ then } (C; \text{while } 0<y \text{ do } C) \text{ else } skip}{s} \to_c \config{\text{if } 0<3 \text{ then } (C; \text{while } 0<y \text{ do } C) \text{ else } skip}{s}}\]
Result:
\[\config{\text{if } 0<3 \text{ then } (a:= a \times y; y:= y - 1; \text{while } 0<y \text{ do } a:= a \times y; y:= y - 1); \text{ else } skip}{(x \mapsto 3, y \mapsto 3, z \mapsto 42, a \mapsto 1)}\]
\subsubsection*{Complete if boolean}
where $C = (a:= a \times y; y:= y - 1)$ and $s = (x \mapsto 3, y \mapsto 3, z \mapsto 42, a \mapsto 1)$:
\[\whilest{W-COND.EXP}{
		\whilest{W-BOOl.LESS.TRUE}{}{\config{0<3}{s} \to_b \config{true}{s}}
	}{\config{\text{if } 0<3 \text{ then } (C; \text{while } 0<y \text{ do } C) \text{ else } skip}{s} \to_c \config{\text{if } true \text{ then } (C; \text{while } 0<y \text{ do } C) \text{ else } skip}{s}}\]
Result:
\[\config{\text{if } true \text{ then } (a:= a \times y; y:= y - 1; \text{while } 0<y \text{ do } a:= a \times y; y:= y - 1); \text{ else } skip}{(x \mapsto 3, y \mapsto 3, z \mapsto 42, a \mapsto 1)}\]
\subsubsection*{Evaluate if}
where $C = (a:= a \times y; y:= y - 1)$ and $s = (x \mapsto 3, y \mapsto 3, z \mapsto 42, a \mapsto 1)$:
\[\whilest{W-COND.TRUE}{}{\config{\text{if } true \text{ then } (C; \text{while } 0<y \text{ do } C) \text{ else } skip}{s} \to_c \config{C; \text{while } 0<y \text{ do } C}{s}}\]
Result:
\[\config{a:= a \times y; y:= y - 1; \text{while } 0<y \text{ do } (a:= a \times y; y:= y - 1)}{(x \mapsto 3, y \mapsto 3, z \mapsto 42, a \mapsto 1)}\]
\subsubsection*{Evaluate Expression a}
where $C = y:= y - 1;  \text{while } 0<y \text{ do } (a:= a \times y; y:= y - 1)$ and $s = (x \mapsto 3, y \mapsto 3, z \mapsto 42, a \mapsto 1)$:
\[\whilest{W-SEQ.LEFT}{
		\whilest{W-ASS.EXP}{
			\whilest{W-EXP.MUL.LEFT}{
				\whilest{W-EXP.VAR}{}{\config{a}{s} \to \config{1}{s}}
			}{\config{a \times y}{s} \to_e \config{1 \times y}{s}}
		}{\config{a := a \times y}{s} \to_c \config{a := 1 \times y}{s}}
	}{\config{a := a \times y; C}{s} \to_c \config{a := 1 \times y; C}{s}}\]
Result:
\[\config{a := 1 \times y; y:= y - 1;  \text{while } 0<y \text{ do } (a:= a \times y; y:= y - 1)}{(x \mapsto 3, y \mapsto 3, z \mapsto 42, a \mapsto 1)}\]
\subsubsection*{Evaluate Expression y}
where $C = y:= y - 1;  \text{while } 0<y \text{ do } (a:= a \times y; y:= y - 1)$ and $s = (x \mapsto 3, y \mapsto 3, z \mapsto 42, a \mapsto 1)$:
\[\whilest{W-SEQ.LEFT}{
		\whilest{W-ASS.EXP}{
			\whilest{W-EXP.MUL.RIGHT}{
				\whilest{W-EXP.VAR}{}{\config{y}{s} \to_e \config{3}{s}}
			}{\config{1 \times y}{s} \to_e \config{1 \times 3}{s}}
		}{\config{a := 1 \times y}{s} \to_c \config{a := 1 \times 3}{s}}
	}{\config{a := 1 \times y; C}{s} \to \config{a := 1 \times 3; C}{s}}\]
Result:
\[\config{a := 1 \times 3; y:= y - 1;  \text{while } 0<y \text{ do } (a:= a \times y; y:= y - 1)}{(x \mapsto 3, y \mapsto 3, z \mapsto 42, a \mapsto 1)}\]
\subsubsection*{Evaluate Multiply}
where $C = y:= y - 1;  \text{while } 0<y \text{ do } (a:= a \times y; y:= y - 1)$ and $s = (x \mapsto 3, y \mapsto 3, z \mapsto 42, a \mapsto 1)$:
\[\whilest{W-SEQ.LEFT}{
		\whilest{W-ASS.EXP}{
			\whilest{W-EXP.MUL}{}{\config{1 \times 3}{s} \to_e \config{3}{s}}
		}{\config{a := 1 \times 3}{s} \to_c \config{a := 3}{s}}
	}{\config{a := 1 \times 3; C}{s} \to_c \config{a := 3; C}{s}}\]
Result:
\[\config{a := 3; y:= y - 1;  \text{while } 0<y \text{ do } (a:= a \times y; y:= y - 1)}{(x \mapsto 3, y \mapsto 3, z \mapsto 42, a \mapsto 1)}\]
\subsubsection*{Assign 3 to a}
where $C = y:= y - 1;  \text{while } 0<y \text{ do } (a:= a \times y; y:= y - 1)$ and $s = (x \mapsto 3, y \mapsto 3, z \mapsto 42, a \mapsto 1)$:
\[\whilest{W-SEQ.LEFT}{
		\whilest{W-ASS.NUM}{}{\config{a := 3}{s} \to_c \config{skip}{s[a \mapsto 3]}}
	}{\config{a := 3; C}{s} \to_c \config{skip;C}{s[a \mapsto 3]}}\]
Result:
\[\config{skip; y:= y - 1;  \text{while } 0<y \text{ do } (a:= a \times y; y:= y - 1)}{(x \mapsto 3, y \mapsto 3, z \mapsto 42, a \mapsto 3)}\]
\subsubsection*{Eliminate Skip}
where $C = y:= y - 1;  \text{while } 0<y \text{ do } (a:= a \times y; y:= y - 1)$ and $s = (x \mapsto 3, y \mapsto 3, z \mapsto 42, a \mapsto 3)$:
\[\whilest{W-SEQ.SKIP}{}{\config{skip;C}{s} \to_c \config{C}{s}}\]
Result:
\[\config{y:= y - 1;  \text{while } 0<y \text{ do } (a:= a \times y; y:= y - 1)}{(x \mapsto 3, y \mapsto 3, z \mapsto 42, a \mapsto 3)}\]
\subsubsection*{Assign 3 to y}
where $C = \text{while } 0<y \text{ do } (a:= a \times y; y:= y - 1)$ and $s = (x \mapsto 3, y \mapsto 3, z \mapsto 42, a \mapsto 3)$:
\[\whilest{W-SEQ.LEFT}{
		\whilest{W-ASS.EXP}{
			\whilest{W-EXP.SUB.LEFT}{
				\whilest{W-EXP.VAR}{}{\config{y}{s} \to \config{3}{s}}
			}{\config{y - 1}{s} \to_e \config{3 - 1}{s}}
		}{\config{y := y - 1}{s} \to_c \config{y := 3 - 1}{s}}
	}{\config{y := y - 1; C}{s} \to_c \config{y := 3 - 1}{s}}\]
Result:
\[\config{y:= 3 - 1;  \text{while } 0<y \text{ do } (a:= a \times y; y:= y - 1)}{(x \mapsto 3, y \mapsto 3, z \mapsto 42, a \mapsto 3)}\]
\subsubsection*{Evaluate Subtraction}
where $C = \text{while } 0<y \text{ do } (a:= a \times y; y:= y - 1)$ and $s = (x \mapsto 3, y \mapsto 3, z \mapsto 42, a \mapsto 3)$:
\[\whilest{W-SEQ.LEFT}{
		\whilest{W-ASS.EXP}{
			\whilest{W-EXP.SUB}{}{\config{3 - 1}{s} \to_e \config{2}{s}}
		}{\config{y := 3 - 1}{s} \to_c \config{y := 2}{s}}
	}{\config{y := 3 - 1;C}{s} \to_c \config{y:= 2;C}{s}}\]
Result:
\[\config{y:= 2;  \text{while } 0<y \text{ do } (a:= a \times y; y:= y - 1)}{(x \mapsto 3, y \mapsto 3, z \mapsto 42, a \mapsto 3)}\]
\subsubsection*{Assign 2 to y}
where $C = \text{while } 0<y \text{ do } (a:= a \times y; y:= y - 1)$ and $s = (x \mapsto 3, y \mapsto 3, z \mapsto 42, a \mapsto 3)$:
\[\whilest{W-SEQ.LEFT}{
		\whilest{W-ASS.NUM}{}{\config{y:= 2}{s} \to_c \config{skip}{s[y \mapsto 2]}}
	}{\config{y:= 2;C}{s} \to_c \config{skip;C}{s[y \mapsto 2]}}\]
Result:
\[\config{skip;\text{while } 0<y \text{ do } (a:= a \times y; y:= y - 1)}{(x \mapsto 3, y \mapsto 2, z \mapsto 42, a \mapsto 3)}\]
\subsubsection*{Eliminate skip}
where $C = \text{while } 0<y \text{ do } (a:= a \times y; y:= y - 1)$ and $s = (x \mapsto 3, y \mapsto 2, z \mapsto 42, a \mapsto 3)$:
\[\whilest{W-SEQ.SKIP}{}{\config{skip;C}{s} \to_c \config{C}{s}}\]
Result:
\[\config{\text{while } 0<y \text{ do } (a:= a \times y; y:= y - 1)}{(x \mapsto 3, y \mapsto 2, z \mapsto 42, a \mapsto 3)}\]
\unfinished
\end{examplebox}
