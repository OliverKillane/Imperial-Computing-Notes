\chapter{Recursion}
We can extend $\Lambda^N$ to include recursion as language $\Lambda^{NR}$.
\begin{itemize}
    \item Definitions can reference their own names, as well as other's names (e.g for mutually recursive functions)
\end{itemize}

\section{Language $\Lambda^{NR}$}
\[\begin{split}
    name & ::= \text{'A string of characters'} \\
    N, M & ::= x \ | \ name \ | \ \lambda x. N \ | \ M \ N \\
    Defs & ::= Defs ; name = M \ | \ Defs ; (rec \ name = M) \ | \ \epsilon \text{ where } M \text{ is closed}\\ 
    Program & ::= Defs : M \\
\end{split}\]
The requirement that $M$ be name-free is removed, and a function labeled $rec$ can be recursive.

\begin{definitionbox}{$\mathsf{Y}$ combinator}
    \[\mathsf{Y} = \lambda f . (\lambda x . f (x \ x)) \ (\lambda x . f (x \ x))\]
    Can be used to encode recursion:
    \[F = C [F] \to \mathsf{Y} \ (\lambda f . C [f]) \]
\end{definitionbox}

\begin{examplebox}{Factorial}
    Write factorial in $\Lambda^{NR}$ given you can use arithmetic and the $Cond$ function. Then encode it using the $\mathsf{Y}$ combinator.
    \tcblower
    \[Factorial = \lambda n . (Cond \ (n == 0) \ 1 \ (n \times (Factorial \ (n - 1))) )\]
    And with the $\mathsf{Y}$ combinator:
    \[Fac = Y . (\lambda f n . Cond \ (n == 0) \ 1 \ (n \times f (n - 1)))\]
\end{examplebox}
We cannot directly translate $\Lambda^{NR}$ to lambda calculus as with $\Lambda^N$, and instead must alter recursive functions to make use of the $\mathsf{Y}$ combinator.
\\
\\ $\mathsf{Y}$ is not typeable under the $\vdash_c$ scheme discussed in these notes, so we must add an extension:
\\ \begin{minipage}[b]{.33\textwidth}
    \[M, N ::= \dots \ | \ \mathsf{Y}\]
    \centerline{Add $Y$ as a special term in the syntax.}
\end{minipage}
\begin{minipage}[b]{.33\textwidth}
    \[\mathsf{Y} \ M \to M (\mathsf{Y} \ M)\]
    \centerline{Add the reduction rule for $\mathsf{Y}$.}
\end{minipage}
\begin{minipage}[b]{.33\textwidth}
    \[\infer{}{\Gamma \vdash \mathsf{Y} : (A \to A) \to A}\]
    \centerline{Add a type assignment rule.}
\end{minipage}

\section{Type Assignment for $\Lambda^{NR}$}
\[\ninfer{$Ax$}{}{\Gamma, x:A; \mathcal{E} \vdash x : A} 
\qquad \ninfer{$\to I$}{\Gamma, x:A ; \mathcal{E} \vdash N : B}{\Gamma ; \mathcal{E} \vdash \lambda x . N : A \to B}
\qquad \ninfer{$\to E$}{\Gamma; \mathcal{E} \vdash P : A \to B \quad \Gamma ; \mathcal{E} \vdash Q : A}{\Gamma ; \mathcal{E} \vdash P \ Q : B}\]
The main $3$ typing rules remain unchanged.
\[\ninfer{$Call$}{}{\Gamma ; \mathcal{E} , name : A \vdash name : S \ A} \qquad \ninfer{$Rec \ Call$}{}{\Gamma; \mathcal{E}, rec \ name: A \vdash name : A}\]
\begin{itemize}
    \item Call remains the same (still just substitutes the definition)
    \item A recursive call is added, however the type cannot be substituted for this as the definition internally relies on the type.
\end{itemize}
\[\ninfer{$Def$}{\mathcal{E} \vdash Defs : \Diamond \quad \emptyset; \mathcal{E} \vdash M : A}{\mathcal{E}, name : A \vdash Defs ; name = M : \Diamond} 
\qquad \ninfer{$Rec \ Def$}{\mathcal{E} \vdash Defs : \Diamond \quad \emptyset; \mathcal{E}, rec \ name: A \vdash M : A}{ \mathcal{E}, name : A \vdash Defs; rec \ name = M : \Diamond }\]
\begin{itemize}
    \item Definitions are no longer name-free and thus we must carry the environment in $Def$.
    \item Recursive calls are typed with the same environment.
\end{itemize}

\[\ninfer{$\epsilon$}{}{\mathcal{E} \vdash \epsilon : \Diamond} \qquad \ninfer{Program}{\mathcal{E} \vdash Defs : \Diamond \quad \Gamma ; \mathcal{E} \vdash M : A}{\Gamma ; \mathcal{E} \vdash Defs : M : A}\]

\section{Principle Types for $\Lambda^{NR}$}

\[\begin{matrix*}[l]
    pp_{\Lambda^{NR}} \ x & \mathcal{E} & = \langle x: \varphi ; \varphi ; \mathcal{E} \rangle \text{ where } \varphi \text{ is fresh} \\

    pp_{\Lambda^{NR}} \ name & \mathcal{E} & = \begin{cases}
        \langle \emptyset ; A ; \mathcal{E} \rangle & (rec \ name : A \in \mathcal{E} ) \\
        \langle \emptyset ; FreshInstance(\mathcal{E} \ name) ; \mathcal{E}  \rangle & (name : A \in \mathcal{E}) \\
    \end{cases} \\
    pp_{\Lambda^{NR}} \ (\lambda x . M) & \mathcal{E} & = \begin{cases}
        \langle \Pi'; A \to P ; \mathcal{E}' \rangle & (\Pi = \Pi', x:A) \\
        \langle \Pi; \varphi \to P ; \mathcal{E}' \rangle & (x \not\in \Pi) \\
    \end{cases} \\
    & & \qquad \begin{matrix*}[l]
        \text{where } & \langle \Pi; P ; \mathcal{E}' \rangle & = pp_{\Lambda^{NR}} \ M \ \mathcal{E} \\
        & \varphi & \text{ is fresh} \\
    \end{matrix*} \\
    pp_{\Lambda^{NR}} \ (M \ N) & \mathcal{E} & = S_2 \circ S_1 \langle \Pi_1 \cup \Pi_2 ; \varphi ; \mathcal{E}'' \rangle \\
    & & \qquad \begin{matrix*}[l]
        \text{where } & \langle \Pi_1 ; P_1 ; \mathcal{E}' \rangle & = pp_{\Lambda^{NR}} \ M \ \mathcal{E} \\
        & \langle \Pi_2 ; P_2 ; \mathcal{E}'' \rangle & = pp_{\Lambda^{NR}} \ N \ \mathcal{E} \\
        & S_1 & = unify \ P_1 \ P_2 \to \varphi \\
        & S_2 & = unifyContexts \ (S_1 \Pi_1) \ (S_1 \Pi_2) \\
        & \varphi & \text{ is fresh} \\
    \end{matrix*}
\end{matrix*}\]
For defs we must modify the $buildEnv$ function to use environments:
\[\begin{matrix*}[l]
    BuildEnv \ (defs; name = M) \ \mathcal{E} & = (BuildEnv \ Defs \ \mathcal{E}), name: A \text{ where } \langle \emptyset ; A ; \mathcal{E} \rangle = pp_{\Lambda^{NR}} \ M \ \mathcal{E} \\
    BuildEnv \ (defs; rec \ name = M) \ \mathcal{E} & = (BuildEnv \ Defs \ \mathcal{E}), name : S \ A \\
    & \qquad \begin{matrix*}[l]
        \text{where } & \langle \emptyset ; A ; \mathcal{E}' \langle & = pp_{\Lambda^{NR}} \ M (\mathcal{E}, rec \ name: \varphi) \\
        & S &= unify \ A \ B \\
        & rec \ name: B & \in \mathcal{E}' \\
        & \varphi & \text{is fresh} \\
    \end{matrix*} \\
    BuildEnv \ \epsilon \ \mathcal{E} & = \mathcal{E} \\
    pp_{\Lambda^{NR}} (Defs ; M)
\end{matrix*}\]
Hence we can now define $pp_{\Lambda^{NR}}$ for $Defs$:
\[pp_{\Lambda^{NR}} \ (Defs; M) = pp_{\Lambda^{NR}} \ M \ \mathcal{E} \text{ where } \mathcal{E} = BuildEnv \ Defs \ \emptyset\]
