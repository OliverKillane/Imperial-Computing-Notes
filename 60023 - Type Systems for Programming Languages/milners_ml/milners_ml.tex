
\newcommand{\mllet}[2]{\text{let } #1 \text{ in } #2}
\newcommand{\mlfix}[2]{\text{fix } #1 . #2}

\chapter{Milner's ML}
\begin{definitionbox}{$\mathcal{L}_{ML}$}
    $\mathcal{L}_{ML}$ is a simple programming language supporting \textit{shallow polymorphic} procedures on a wide variety of objects.
    \begin{itemize}
        \item It is an extension of $\lambda$-calculus
        \item Adds a construct for expressing recursion
        \item Adds a construct for expressing sub-terms can be used in different ways.
    \end{itemize}
\end{definitionbox}

A new type-assignment algorithm is paired with $\mathcal{L}_{ML}$ called $\mathcal{W}$:
\begin{itemize}
    \item \textit{Semantically Sound} - all typed programs are correct.
    \item \textit{Syntactically Sound} - if $\mathcal{W}$ accepts a program, then it is well-typed.
\end{itemize}

\section{The ML Type Assignment System}
$ML$ expressions are of the form:
\[E ::= x \ | \ c \ | \ \lambda x.E \ | \ E_1 \ E_2 \ | \ \mllet{x = E_1}{E_2} \ | \ \text{fix }g . E\]
where:
\begin{itemize}
    \item $x$ is bound over $E_2$ in $\text{let }x = E_1 \text{ in } E_2$
    \item $g$ is bound over $E$ in $\text{fix }g . E$
    \item $c$ is a term constant, such as a number, character or operator
\end{itemize}
\subsection{Term Substitution}
Term substitution is defined as with the following rules:
\\ \begin{minipage}[t]{.5\textwidth}
    \[\begin{split}
        x[E/x] &= x \\
        y[E/x] &= y \qquad (y \neq x) \\
    \end{split}\]
    \centerline{Basic substitution of variables.}
\end{minipage}
\begin{minipage}[t]{.5\textwidth}
    \[\begin{split}
        (\lambda y . E')[E/x] &= \lambda y.(E'[E/x]) \\
        (E_1 \ E_2)[E/x] &= E_1[E/x] \ E_2[E/x] \\
    \end{split}\]
    \centerline{Substitution of sub-terms.}
\end{minipage}
\vspace{5mm}
\[\begin{split}
    (\mllet{x = E_1}{E_2})[E/x] &= \mllet{y = E_1[E/x]}{E_2[E/x]} \\
    (\mlfix{g}{E'})[E/x] &= \mlfix{g}{E'[E/x]} \\
\end{split}\]
\centerline{let statements and fixed point (for recursion)}
Note that \textit{barendregt's convention} assumed here.
\begin{itemize}
    \item The \textit{let} construction is added to cover cases where $(\lambda x . E_1) E_2$ is not typeable but where the contraction $E_1[E_2/x]$ is typeable.
    \item The \textit{fix} construction introduces model recursion. It is not a combinator, but rather another abstraction mechanism (e.g like $\lambda . $).
\end{itemize}

\subsection{Reduction}
Reduction on $\mathcal{L}{ML}$ is $\to_{ML}$ and is defined as an extension of $\to_\beta$, with the additional rules:
\[\begin{split}
    \mllet{x_1 = E_1}{E_2} & \to_{ML} E_2[E_1/x] \\
    \mlfix{g}{E} & \to_{ML} E[(\mlfix{g}{E})/g] \\
\end{split}\]
We also add some contextual rules.
\[E \to_{ML} E' \Rightarrow \begin{cases}
    \mllet{x = E}{E_2} &\to_{ML} \mllet{x = E'}{E_2} \\
    \mllet{x = E_1}{E} &\to_{ML} \mllet{x = E_1}{E'} \\
    \mlfix{g}{E} & \to_{ML} \mlfix{g}{E'} \\
\end{cases}\]
Under reduction both $\mllet{x = E_2}{E_1}$ and $(\lambda x . E_1) \ E_2$ are \textit{reducible expressions} and both reduce to $E_1[E_2/x]$
\begin{itemize}
    \item $(\lambda x . E_1) \ E_2$ semantically interpreted as a function with an operand $x$
    \item $\mllet{x = E_2}{E_1}$ interpreted as a substitution.
\end{itemize}
Type assignment treats both differently.

\subsection{Type Assignment}
The set of types is defined similarly to with curry types ($\mathcal{T}_c$).
\begin{itemize}
    \item Extended with type constants $'C$ that includes $\text{int}, \text{bool}, \dots$.
    \item Ranged over by type $A, B, \dots$ much like with $\mathcal{T}_c$.
    \item Types can be \textit{quantified}, creating \textit{generic types} / \textit{type schemes} ranged over by $\sigma, \tau, \dots$.
\end{itemize}
\[\begin{matrix*}[l]
    A, B & ::= \varphi \ | \ c \ | \ (A \to B) & (\text{basic types}) \\
    \sigma , \tau & ::= A \ | \ (\forall \varphi . \tau) & (\text{polymorphic types}) \\
\end{matrix*}\]
Types of the form $\forall \varphi . \tau$ are called \textit{quantified types}.
\begin{itemize}
    \item $(\forall \varphi_1 . (\forall \varphi_2 . \dots (\forall \varphi_n . A)\dots))$ is abbreviated by $\forall \overset{\rightharpoonup}{\varphi} . A$
    \item $\varphi$ is bound in $\forall \varphi . \tau$
    \item Free and bound type variables can be defined just as with variables in $\lambda$-calculus, but must have names kept separate.
\end{itemize}

\textit{ML type substitution} is defined as:
\\ \begin{minipage}[b]{.5\textwidth}
    \[\begin{matrix*}[l]
        (\varphi \mapsto C) & \varphi & = C \\
        (\varphi \mapsto C) & c & = c \\
        (\varphi \mapsto C) & \varphi' & = \varphi' & (\varphi' \neq \varphi) \\
        (\varphi \mapsto C) & A \to B & = ((\varphi \mapsto C) A) \to ((\varphi \mapsto C) B) \\
    \end{matrix*}\]
    \centerline{Basic type substitutions}
\end{minipage}
\begin{minipage}[b]{.5\textwidth}
    \[\begin{matrix*}[l]
        (\varphi \mapsto C) & \forall \varphi' . \psi & = \forall \varphi' .((\varphi \mapsto C) \psi) \\
    \end{matrix*}\]
    \centerline{Quantified types}
\end{minipage}
\vspace{2mm}
Unification is also extended with type constants as:
\[\begin{matrix*}[l]
    unify & \varphi & c & = (\varphi \mapsto c) \\
    unify & c & \varphi & = unify \ \varphi \ c \\
    unify & c & c & = Id_S \\
\end{matrix*}\]
\begin{itemize}
    \item Here a unification of all other cases including a type constant will fail (e.g cannot unify $\text{int}$ and $\text{bool}$)
    \item Types are considered modulo a kind of $\alpha$-conversion (similar to barendregt's convention - avoid type variable capture)
    \item As $\varphi'$ is bound in $\forall \varphi' . \psi$ we can assume in $(\varphi \mapsto C) \ \forall \varphi' . \psi$ we have $\varphi \neq \varphi'$ and $\varphi' \not\in fv(C)$.
    \item As we can have free type variables, the set of types occurring in $\forall \varphi_1 \dots \forall \varphi_n . A$ is not necessarily $\{\varphi_1, \dots , \varphi_n\}$.
\end{itemize}
\[\overline{\Gamma} \ A = \forall \overset{\rightharpoonup}{\varphi} . A\]
$\overset{\rightharpoonup}{\varphi}$ appear free in $A$, but are not in the context of $A$.
\\
\\ For the inference system expressing type assignment, we include a function $\nu$ which maps constants to their type (e.g a constant type such as Char, Int or a closed polymorphic type).
\\ \begin{minipage}[b]{.5\textwidth}
    \[\ninfer{$Ax$}{}{\Gamma, x:\tau \vdash x : \tau}\]
    \centerline{Basic substitution of free variable}
\end{minipage}
\begin{minipage}[b]{.5\textwidth}
    \[\ninfer{$\mathcal{C}$}{}{\Gamma \vdash c : \nu \ c}\]
    \centerline{Substituting constants}
\end{minipage}
\vspace{2mm}
\\ \begin{minipage}[b]{.5\textwidth}
    \[\ninfer{$\to I$}{\Gamma, x:A \vdash E : B}{\Gamma \vdash \lambda x . E : A \to B}\]
\end{minipage}
\begin{minipage}[b]{.5\textwidth}
    \[\ninfer{$\to E$}{\Gamma \vdash E_1 : A \to B \quad \Gamma \vdash E_2 : A}{\Gamma \vdash E_1 \ E_2 : B}\]
\end{minipage}
\vspace{2mm}
\\ \begin{minipage}[b]{.5\textwidth}
    \[\ninfer{let}{\Gamma \vdash E_1 : \tau \quad \Gamma, x: \tau \vdash E_2 : B}{\mllet{x = E_1}{E_2 : B}}\]
\end{minipage}
\begin{minipage}[b]{.5\textwidth}
    \[\ninfer{fix}{\Gamma, g:A \vdash E : A}{\Gamma \vdash \mlfix{g}{E} : A}\]
\end{minipage}
\vspace{2mm}
\\ \begin{minipage}[b]{.5\textwidth}
    \[\ninfers{$\forall I$}{\Gamma \vdash E : \tau}{\Gamma \vdash E : \forall \varphi . \tau}{\varphi \text{ not free in } \Gamma}  \]
\end{minipage}
\begin{minipage}[b]{.5\textwidth}
    \[\ninfer{$\forall E$}{\Gamma \vdash E : \forall \varphi . \tau}{\Gamma \vdash E : \tau [A / \varphi]}\]
\end{minipage}
\vspace{5mm}
\\ Quantification is introduced to model substitution operations on types, rasther than replacing all type variables at once.
\begin{itemize}
    \item $\forall \varphi : \tau$ - all occurrences of type variable $\varphi$ can be replaced by some basic type.
    \item The side condition on $\forall I$ ensures that the type variables used do not also occur in the context (there is no reference back to the context).
\end{itemize}
We can model the substitution of $\varphi$ in $A$, by type $B$ as $(\varphi \mapsto B) \ A$.
\[\ainfer{$\forall E$}{\ainfer{$\forall I$}{\emptyset \vdash_{ML} E : A}{\emptyset \vdash_{ML} E : \forall \varphi . A}}{\emptyset \vdash_{ML} E : A [B/\varphi]}\]
The let construct corresponds to definitions in $A^{NR}$.
\begin{itemize}
    \item Can occur anywhere within a term.
    \item Given $\mllet{x = E_1}{E_1}$, $E_1$ does not need to be a \textit{closed-term}, so it is possible to define terms that are \textit{partially-polymorphic} (a term of type $\forall \overset{\rightharpoonup}{\varphi} . A$ where $A$ contains free type variables).
    \item When applying $\forall I$ only the type variable that we attempt to bind must not occur in the context.
\end{itemize}
To allow for recursion to be typed, the syntax for fix is added.
\begin{itemize}
    \item Previously we have seen $\mathbf{Y}$ added as a typed constant.
    \item It is also possible to solve this by defining \textit{letrec} as a combination of \textit{let} and \textit{fix} ($\text{letrec } g = \lambda x . E_1 in E_2$)
\end{itemize}
\subsection{Lemmas for Type Assignment}
\subsubsection{Free Variables}
\[(\Gamma \vdash_{ML} E : \tau \land x \in fv(E)) \Rightarrow \exists \sigma . [x: \sigma \in \Gamma]\]
All free variables in some expression must have a type in the context.
\subsubsection{Weakening}
\[(\Gamma \vdash_{ML} E : \tau \land \forall x:\sigma \in \Gamma' [x:\sigma \in \Gamma \lor x \not\in (fv(E) \cup bv(E)) ]) \Rightarrow \Gamma' \vdash_{ML} E : \tau\]
Given some context $\Gamma$, any context that extends $\Gamma$ without adding any of $E$'s variables is equivalent.
\subsubsection{Thinning}
\[(\Gamma x:\sigma \vdash_{ML} E : \tau \land x \not\in fv(E)) \Rightarrow \Gamma \vdash_{ML} E : \tau \]
We can remove variables that are not free in $E$ from the context, and the context will still be able to type $E$.

\subsubsection{Generation}
\begin{definitionbox}{$>_\Gamma$}
    \[\Gamma \vdash_{ML} E : A[B / \varphi]\]
    The smallest reflexive and transitive relation such that:
    \[\begin{split}
        \rho & >_\Gamma \forall \varphi . \rho \qquad (\varphi \text{ is not free in }\Gamma\text{ and not bound in }\rho)\\
        \forall \varphi . \rho & >_\Gamma \rho[B / \varphi] \\
    \end{split}\]
    Where there are no free $\varphi'$ in $A$.
    \begin{itemize}
        \item If $\sigma >_\Gamma \tau$ then $\tau$ is a \textit{generic instance} of $\sigma$.
        \item Each context $\Gamma$ induces a new relation.
        \item This relation represents applying the $\forall I$ and $\forall E$ steps.
    \end{itemize}
    \[\Gamma \vdash_{ML} E : \sigma \land \sigma >_{\Gamma} \tau \Rightarrow \Gamma_{ML} E : \tau\]
\end{definitionbox}

\[\begin{matrix*}[l]
    (1) & \Gamma \vdash_{ML} & x : \sigma & \Rightarrow & \exists x:\tau \in \Gamma . & \tau >_\Gamma \sigma \\
    \\
    (2) & \Gamma \vdash_{ML} & \lambda x . E : \sigma & \Rightarrow & \exists A,B . & \ \ \quad  \Gamma, x:A \vdash_{ML} E : B  \\
    & & & & & \land \quad \sigma = \forall \overset{\rightharpoonup}{\varphi_i} . A \to B \\
    & & & & & \land \quad A \to B >_\Gamma \sigma \\
    \\ 
    (3) & \Gamma \vdash_{ML} & E_1 \ E_2 : \sigma & \Rightarrow & \exists A,B . & \ \ \quad \Gamma \vdash_{ML} E_1 : A \to B \\
    & & & & & \land \quad  \Gamma \vdash_{ML} E_2 : A \\
    & & & & & \land \quad  B >_{\Gamma} \sigma \\
    \\
    (4) & \Gamma \vdash_{ML} & \mlfix{g}{E} : \sigma & \Rightarrow & \exists A . & \ \quad \Gamma , g: A \vdash_{ML} E : A \\
    & & & & & \land \quad \sigma = \sigma = \forall \overset{\rightharpoonup}{\varphi_i} . A \\
    & & & & & \land \quad A >_\Gamma \sigma \\
    \\
    (5) & \Gamma \vdash_{ML} & \mllet{x = E_1}{E_2} : \sigma & \Rightarrow & \exists A, \tau & \ \ \quad \Gamma  , x: \tau \vdash_{ML} E_2 : A \\
    & & & & & \land \quad \Gamma \vdash_{ML} E_1 : \tau \\
    & & & & & \land \quad A >_\Gamma \sigma \\
\end{matrix*}\]

\begin{sidenotebox}{System F}
    The ML type assignment is a restriction on the \textit{polymorphic type discipline} (System F).
    \begin{itemize}
        \item In the ML type assignment covered in these notes, $\forall$ occurs outside of a type (\textit{shallow polymorphism}). It is also decidable.
        \item In System F $\forall$ is a general type constructor, so $A \to \forall \varphi . B$ is a valid type. It is not decidable.
    \end{itemize}
\end{sidenotebox}

\begin{examplebox}{Complex Types}
    Type $\mllet{i = \lambda x . x}{i \ i}$.
    \tcblower
    \[\ainfer{let}{
        \ainfer{$\forall I$}{
            \ainfer{$\to I$}{
                \ainfer{$Ax$}{
                }{
                    x : \varphi \vdash x : \varphi
                }
            }{
                \emptyset \vdash \lambda x . x : \varphi \to \varphi
            }
        }{
            \emptyset \vdash \lambda x . x : \forall \varphi . \varphi \to \varphi
        } \qquad \ainfer{$\to E$}{
            (1) \qquad (2)
        }{
            i: \forall \varphi . \varphi \to \varphi \vdash i \ i : A \to A
        }
    }{\emptyset \vdash \mllet{i = \lambda x . x}{i \ i} : A \to A}\]
    where:
    \[(1) = \ainfer{$\forall E$}{
        \ainfer{$Ax$}{
        }{
            i: \forall \varphi . \varphi \to \varphi \vdash i : \forall \varphi . \varphi \to \varphi
        }
    }{
        i: \forall \varphi . \varphi \to \varphi \vdash i : (A \to A) \to A \to A
    }\]
    \[(2) = \ainfer{$\forall E$}{
        \ainfer{$Ax$}{
        }{
            i: \forall \varphi . \varphi \to \varphi \vdash i : \forall \varphi . \varphi \to \varphi
        }
    }{
        i: \forall \varphi . \varphi \to \varphi \vdash i : A \to A
    }\]
\end{examplebox}

\begin{examplebox}{Addition}
    Express Addition in ML.
    \tcblower
    We can use type constants by defining then in $\nu$:
    \[\nu \ x = \begin{cases}
        Num \to Num & x = Succ \\
        Num \to Num & x = Pred \\
        Num \to Bool & x = IsZero \\
        \forall \varphi . Bool \to \varphi \to \varphi \to \varphi & x = Cond \\
        \vdots & \vdots \\
    \end{cases}\]
    We can define it recursively as:
    \[Add = \lambda x y . Cond \ (IsZero x) \ y \ (Succ \ (Add \ (Pred x) y))\]
    This recursion can be implemented in ML using fix.
    \[Add = \mlfix{a}{\lambda x y . Cond \ (IsZero x) \ y \ (Succ \ (a \ (Pred x) y))}\]
\end{examplebox}

\section{Milner's $\mathcal{W}$}
\begin{definitionbox}{Milner's Type Assignment Algorithm}
    Milner's $\mathcal{W}$ is a type assignment algorithm for ML.
    \begin{itemize}
        \item It has a \textit{principle type property} - given any $\Gamma$ and $E$ there is a principle type computed by $\mathcal{W}$.
        \item Its does not have the \textit{principle pair property} as if $\Gamma, x:\tau \vdash_{ML} E : A$ may exist, but $\lambda x. E$ may not be typeable.
        \item Type assignment is decidable.
    \end{itemize}
    It is complete, given some $E$, contexts $\Gamma$ and $\Gamma'$ and type $A$:
    \[\Gamma' \text{ is an instance of } \Gamma \land \Gamma' \vdash_{ML} E : A \Rightarrow \mathcal{W} \ \Gamma \ E = \langle S, B \rangle \land \exists \ S' . \ [\Gamma' = S' (S \ \Gamma) \land S' (S \ B) >_{\Gamma'} A]\]
    It is also sound:
    \[\forall \ E . \ [\mathcal{W} \ \Gamma \ E = \langle S, A \rangle \Rightarrow S \ \Gamma \vdash_{ML} E : A]\]
\end{definitionbox}

\subsection{Basic Cases}
\begin{minipage}{.5\textwidth}  
\[\begin{matrix*}[l]
    \mathcal{W} \ \Gamma \ c & = \langle id, B \rangle \\
    & \begin{matrix*}[l]
        \text{where } & \nu \ c & = \forall \overset{\rightharpoonup}{\varphi} . A \\
        & B & = A [\overset{\rightharpoonup}{\varphi' / \varphi}] \\
        & \text{all } \varphi' & \text{ are fresh} \\
    \end{matrix*} \\
    \\
    \mathcal{W} \ \Gamma \ (\lambda x . E) & = \langle S , S \ (\varphi \mapsto A) \rangle \\
    & \begin{matrix*}[l]
        \text{where } & \langle S, A \rangle &= \mathcal{W} \ (\Gamma, x:\varphi) \ E \\
        & \varphi & \text{is fresh} \\
    \end{matrix*} \\
\end{matrix*}\]
\end{minipage}
\vline
\begin{minipage}{.5\textwidth}
\[\begin{matrix*}[l]
    \mathcal{W} \ \Gamma \ x & = \langle id, B \rangle \\
    & \begin{matrix*}[l]
        \text{where } & x:\forall \overset{\rightharpoonup}{\varphi} . A & \in \Gamma \\
        & B & = A [\overset{\rightharpoonup}{\varphi' / \varphi}] \\
        & \text{all } \varphi' & \text{are fresh} \\
    \end{matrix*} \\
\end{matrix*}\]
\end{minipage}

\subsection{Let Construct}

\[\begin{matrix*}[l]
    \mathcal{W} \ \Gamma \ (\mllet{x = E_1}{E_2}) & = \langle S_2 \circ S_1, B \rangle \\
    & \begin{matrix*}[l]
        \text{where } & \langle S_1, A \rangle & = \mathcal{W} \ \Gamma \ E_1 \\
        & \langle S_2, B \rangle & = \mathcal{W} \ (S_1 \Gamma, x:\sigma) \ E_2 \\
        & \sigma & = \overline{S_1\Gamma} A \\
    \end{matrix*} \\
\end{matrix*}\]
\begin{enumerate}
    \item Get the type and substitutions for $E_1$ given the context $\Gamma$
    \item Get the type and substitutions for $E_2$, the context needs to have $E_1$'s substitutions applied, we add in a new variable $x$ (it will be free in $E_2$) and give it a $\forall$ type that uses no type variables already bound in $S_1 \ \Gamma$.
    \item The resulting type for $E_2$ is the type of the whole term, we must compose the substitutions for $E_1$ and $E_2$.
\end{enumerate}

\subsection{Fix Construct}

\[\begin{matrix*}[l]
    \mathcal{W} \ \Gamma \ (\mlfix{g}{E}) & = \langle S_2 \circ S_1, S_2 \ A \rangle \\
    & \begin{matrix*}[l]
        \text{where } & \langle S_1, A \rangle &= \mathcal{W} \ (\Gamma, g:\varphi) \ E \\
        & S_2 & = unify \ (S_1 \varphi) \ A \\
        & \varphi & \text{is fresh} \\
    \end{matrix*} \\
\end{matrix*}\]
\begin{enumerate}
    \item $g$ must have the same type as $E$ (recursion, the inner call has the same type as the outer), hence to compute the pair for $E$ we add $g$ to the context with fresh type variable $\varphi$
    \item We then get a substitution $S_1$ and type $A$, we must unify this with the type of $g$ (with $S_1$ applied) to type the whole term.
\end{enumerate}

\subsection{Application}

\[\begin{matrix*}[l]
    \mathcal{W} \ \Gamma \ (E_1 \ E_2) & = \langle S_3 \circ S_2 \circ S_1, S_2 \varphi \rangle \\
    & \begin{matrix*}[l]
        \text{where } & \langle S_1, A \rangle & = \mathcal{W} \ \Gamma \ E_1 \\
        & \langle S_2, B \rangle & = \mathcal{W} \ (S_1 \ \Gamma) \ E_2 \\
        & S_3 & = unify \ (S_2 \ A) \ (B \to \varphi) \\
        & \varphi & \text{is fresh} \\
    \end{matrix*} \\
\end{matrix*}\]
\begin{enumerate}
    \item First the type of $E_1$ is computed ass type $A$, with substitutions $S_1$.
    \item Next we get the type of $E_2$, first applying the substitution $S_1$ to the context $\Gamma$.
    \item We now have $E_1 : A$ and $E_2 : B$. $A$ must be equal to some type $B \to \varphi$ ($E_1$ is a function taking $E_2$ as input), hence we unify $A$ with $B \to \varphi$
\end{enumerate}

\section{Polymorphic Recursion}
Mycroft generalised Milner's system in an attempt to improve typing for recursively defined objects.
\begin{minted}{Haskell}
map f ls      = if null ls then ls else cons (f (head ls)) (map f (tail ls))
squarelist ls = map (\x -> x^2) ls
\end{minted}
In $\Lambda^{NR}$ this would be defined as:
\[\begin{matrix*}[l]
    \vdots & (\text{definitions of head, tail, cons, etc}) \\
    map & = \lambda f \ ls . Cond \ (null \ ls) \ ls \ (cons \ (f \ (head \ ls)) \ (map \ f \ (tail \ ls))) \\
    squarelist & = \lambda ls . map \ (\lambda x . mul \ x \ x) \ ls \\
\end{matrix*}\]
The name $squarelist$ could then be used in a program.
\\
\\ In ML there is no check to see if functions are independent or mutually recursive, so all definitions must be done in a single step. 
Hence we can extend $\mathcal{L}_{ML}$ with a pairing function $\langle \ . \ , \ . \ \rangle$:
\[\mllet{\langle map, squarelist \rangle = \mlfix{\langle m, s \rangle}{\langle \lambda f \ ls . Cond \ (null \ ls) \ l \ (cons \ (f \ (head \ ls)) \ (m \ f \ (tail \ ls))), \lambda ls . m \ (\lambda x . mul \ x \ x) \ ls\rangle}}{\dots}\]
However we still have a type assignment issue, $\mathcal{W}$ will get the following types:
\[\begin{matrix*}[l]
    map & :: (num \to num) \to [num] \to [num] \\
    squarelist & :: [num] \to [num] \\
\end{matrix*}\]
The definition of map has the type:
\[map :: \forall \varphi_1 \varphi_2 . (\varphi_1 \to \varphi_2) \to [\varphi_1] \to [\varphi_2]\]

For $\mlfix{g}{E}$ milner's $\mathcal{W}$ unifies the type of $E$ and $g$, this results in the second type of map not being found by type inference.
\\
\\ One way to avoid this problem is to treat the term as a single definition.
\[\mllet{map = \mlfix{m}{\lambda f \ ls . Cond \ (null \ ls) \ ls \ (cons \ (f \ (head \ ls)) \ (m \ f \ (tail \ ls)))}}{\mllet{squarelist = \lambda ls . map \ (\lambda x . mul \ x \ x) \ ls}{\dots}}\]

Instead in Mycroft's system the fix rule is altered.
\\ \begin{minipage}[b]{.5\textwidth}
    \[\ninfer{fix}{\Gamma, g:A \vdash E : A}{\Gamma \vdash \mlfix{g}{E} : A}\]
    \centerline{\textbf{Milner's}}
\end{minipage}
\begin{minipage}[b]{.5\textwidth}
    \[\ninfer{fix}{\Gamma, g: \tau \vdash_{MYC} E : \tau}{\Gamma \vdash_{MYC} \mlfix{g}{E} : \tau}\]
    \centerline{\textbf{Mycroft's}}
\end{minipage}
Hence the derivation rule allows for type-schemes (the $\tau$) which means different curry types (e.g $A$) may be used.


\subsection{Mycroft-Style Assignment for $\Lambda^{NR}$}
The rules for $\epsilon$, Call, Rec Call, Def and Rec Def can be replaced by the rules:
\[\ninfer{$\epsilon$}{}{\mathcal{E} \vdash \epsilon : \diamond} \qquad \ninfer{Call}{}{\Gamma;\mathcal{E},name:A \vdash name : S \ A} \qquad \ninfer{Defs}{\mathcal{E}, name : A \vdash Defs : \diamond \quad \emptyset; \mathcal{E}, name: A \vdash M : A}{\mathcal{E}, name : A \vdash Defs ; name = M : \diamond}\]
With the principle pair algorithm as:
\[\begin{matrix*}[l]
    pp_{\Lambda^{RN}} \ x \ \mathcal{E} & = \langle x:\mathcal{E}; \mathcal{E} \rangle \text{ where } \mathcal{E} \text{ is fresh} \\
    \\
    pp_{\Lambda^{RN}} \ name \ \mathcal{E} & = \langle \emptyset; FreshInstance(\mathcal{E} \ name) \rangle \\
    \\
    pp_{\Lambda^{RN}} \ (\lambda x . M) \ \mathcal{E} & = \begin{cases}
        \langle \Pi'; A \to P \rangle & (\Pi = \Pi',x:A) \\
        \langle \Pi; \varphi \to P \rangle & (x \not\in \Pi) \\
    \end{cases} \\
    & \begin{matrix*}[l]
        \qquad \text{where } & \langle \Pi ; P \rangle & = pp_{\Lambda^{RN}} \ M \ \mathcal{E} \\
        & \varphi & \text{is fresh} \\
    \end{matrix*} \\
    \\
    pp_{\Lambda^{RN}} \ (M \ N) \ \mathcal{E} & = S_2 \circ S_1 \langle \Pi_1 \cup \Pi_2; \mathcal{E} \rangle \\
    & \begin{matrix*}[l]
        \text{where } & \langle \Pi_1 ; P_1 \rangle & = pp_{\Lambda^{RN}} \ M \ \mathcal{E} \\
        & \langle \Pi_2 ; P_2 \rangle & = pp_{\Lambda^{RN}} \ M \ \mathcal{E} \\
        S_1 & = unify P_1 \ (P_2 \to \varphi) \\
        s_2 & = unifyContexts \ (S_1 \ \Pi_1) \ (S_1 \ \Pi_2) \\
        \varphi & \text{is fresh} \\
    \end{matrix*} \\
    \\
    pp_{\Lambda^{RN}} \ (Defs ; M) \ \mathcal{E} & = \begin{cases}
        pp_{\Lambda^{RN}} \ M \ \mathcal{E} & \text{if } CheckEnv \ Defs \ \mathcal{E} \\
        \text{untypeable} & otherwise \\
    \end{cases} \\
    & \begin{matrix*}[l]
        \qquad \text{where } & CheckEnv \ (Defs;name = M) \ \mathcal{E} & = (CheckEnv \ Defs \ \mathcal{E}) \land (\mathcal{E} \ name) = P \\
        & & \qquad \text{where} \langle \emptyset, P \rangle = pp_{\Lambda^{RN}} \ M \ \mathcal{E} \\
        & CheckEnv \ \epsilon \ \mathcal{E} & = true \\
    \end{matrix*}
\end{matrix*}\]

\subsection{Milner's and Mycroft's System's Differences}
As Mycroft's system is an extension of Milner's:
\[\text{Typeable in Milner's} \Rightarrow \text{Typeable in Mycroft's}\]
\begin{itemize}
    \item Many terms are typeable in Mycroft's but not in Milner's
    \item Some terms can be given more general type in Mycroft's than Milner's
\end{itemize}
The key difference between the systems is that in Milner's recursive calls use the same curry type, but in Mycroft's these can be a more general type, this allows for polymorphic recursion.
\begin{examplebox}{Typeability}
    Create a term typeable in Mycroft's System but not in Milner's.
    \tcblower
    \[\mlfix{g}{(\lambda (ab.a) \ (g \ \lambda c . c) \ (g \ \lambda de.d ))}\]
    We can write this as:
    \begin{minted}{Haskell}
        func :: a -> b
        func = (\a b -> a) (func (\c -> c))  (func (\lambda d e -> d))
        -- or more idomatically
        func' = const (func' id) (func' const)
        
        -- in execution this looks like:
        func x
          = const (func' id) (func' const) x
          = func' id x
          = const (func' id) (func' const) id x
          = func' id id x
            ...
          = func' id id ... id x
    \end{minted}
    This is not typeable in Milner's as here $g$ effectively has two types. Mycroft's allows this as both types can come from the polymorphic type $\forall \varphi_1 \varphi_2 . \varphi_1 \to \varphi_2$.
\end{examplebox}
