\chapter{Curry Type Assignment}
Type assignment follows the syntactic structure of terms. For example $\lambda x . M$ will be of the form $A \to B$ where the input $x$ is of type $A$, and $M$ is of type $B$.
\\
\\ $\mathcal{T_C}$ is the set of \textit{types}.
\begin{itemize}
	\item This is ranged over by $A, B \dots$ and defined over the set of \textit{type variables} $\Phi$.
	\item The set of \textit{type variables} $\Phi$ is ranged over by $\varphi$
\end{itemize}
\[A, B ::= \varphi \ | \ (A \to B) \]
A type can be either some type variable (some type e.g Int), or a function converting one type to another.
\begin{definitionbox}{Statement}
	An expression of the form $M : A$ where $M \in \Lambda$ and $A \in \mathcal{T}_c$.
	\begin{itemize}
		\item $M$ is the \textit{subject}
		\item $A$ is the \textit{predicate}
	\end{itemize}
\end{definitionbox}
\begin{definitionbox}{Context}
	A context $\Gamma$ is a set of statements with distinct variables as subjects.
	\begin{itemize}
		\item $\Gamma , x:A$ is shorthand for $\Gamma \cup \{x: A\}$ where $x$ does not occur as a subject in $\Gamma$ (variables must be distinct).
		\item $x:A$ is shorthand for $\emptyset, x:A$.
		\item $x \in \Gamma$ is shorthand for $\exists A \in \mathcal{T}_C . [x:A \in \Gamma]$, likewise, if $x$ is not typed in the context we use $x \not\in \Gamma$.
	\end{itemize}
	For example:
	\[\Gamma_{\text{my context}} = \{x : A, y : B, c: B\}\]
\end{definitionbox}

$\to$ is used for function types, it is right associative, so:
\[(A \to B) \to C \to D \equiv (A \to B) \to (C \to D)\]

\subsection{Curry Type Assignment}
\[\ninfer{$Ax$}{}{\Gamma, x:A \vdash_C x: A} \qquad \ninfers{$\to I$}{\Gamma, x: A \vdash_C M : B}{\Gamma \vdash_C \lambda x. M : A \to B}{x \not\in \Gamma} \qquad \ninfer{$\to E$}{\Gamma \vdash_C M_1 : A \to B \quad \Gamma \vdash_C M_2 : A}{\Gamma \vdash_c M_1 M_2 : B}\]
\vspace{5mm}

\begin{itemize}
	\item {We can extend barendregt's convention to ommit the side-condition on $\to I$ by adding the assertion that:
	      \[\Gamma \vdash M : A \text{ we ensure } \forall x \in bv(M) . [x \not\in \Gamma]\]
	      }
	\item {The definition provided is \textit{sound}:
	      \[(\Gamma \vdash_c M : A) \land (M \to^*_\beta N) \Rightarrow \Gamma \vdash_C N : A \]
	      }
\end{itemize}

Some terms are not typeable under this definition, as self-application is not possible:
\begin{itemize}
	\item $\lambda x . x \ x$ is not typeable, neither is $\lambda f. (\lambda x. f(x \ x)) (\lambda x. f(x \ x))$
	\item Type assignment rules do not cover approximants, and hence they are not typeable.
\end{itemize}

\begin{examplebox}{Self Application}
	Is it possible to type \textit{self-application} $x \ x$?
	\tcblower
	We can attempt to use the inference system, however run into a contradiction:
	\[\ainfer{$\to E$}{
			\ainfer{$Ax$}{}{\Gamma, x: A \to B \vdash_C x : A \to B} \qquad \ainfer{$Ax$}{}{\Gamma, x : A \vdash_C x : A}
		}{\Gamma, x: ? \vdash_C x \ x : B}\]
	Hence we need a type such that $A \to B = A$.
\end{examplebox}

\subsection{Important Lemmas For Type Assignment}
\subsubsection{Term Substitution}
\[\exists C . [(\Gamma, x:C \vdash_C M:A) \land (\Gamma \vdash_C N : C)] \Rightarrow \Gamma \vdash_C M[N/x] : A\]

\subsubsection{Free Variables}
\[\Gamma \vdash_C M : A \land x \in fv(M) \Rightarrow \exists B \in \mathcal{T}_C . [x : B \in \Gamma]\]
All free variables in $M$ are typed.
\subsubsection{Weakening}
\[\Gamma \vdash_C M : A \land \Gamma' \text{ is such that } \forall x: B \in \Gamma' . [x:B \in \Gamma \lor (x \not\in fv(M) \land x \not\in bv(M)) \Rightarrow \Gamma' \vdash_C M : A]\]
We can create a new context $\Gamma'$ that types variables $x,y,z, \dots$. If for every variable in the context $\Gamma'$ it is either not in $\Gamma$, or is in $\Gamma$  with the same type, then we can use $\Gamma'$ to type $M$.

\subsubsection{Thinning}
\[\Gamma, x:B \vdash_C M : A \land x \not\in fv(M) \Rightarrow \Gamma \vdash_C M : A\]
If a variable is not free in $M$, then we do not need a type for it.

\section{Principle Type Property}
Principle type theory expresses the idea that a whole family of types could be assigned to a term, however only one is the \textit{principle type}.
\begin{definitionbox}{Type Substitution}
	\[(\varphi \mapsto C) : \mathcal{T}_C \to \mathcal{T}_C \text{ where } \varphi \text{ is a type variable and } C \in \mathcal{T}_C\]
	Substitution is defined by:
	\[\begin{matrix*}[l]
			(\varphi \mapsto C) & \varphi & = C \\
			(\varphi \mapsto C) & \varphi' & = \varphi' & (\varphi \neq \varphi') \\
			(\varphi \mapsto C) & A \to B &= ((\varphi \mapsto C) \ A) \to ((\varphi \mapsto C) \ B) \\
		\end{matrix*}\]
	Here $(\varphi \mapsto C)$ is a substitution substituting the type variable $\varphi$ for the type $C$
	\[S_1 \circ S_2 \text{ means } S_1 \circ S_2 \ A = S_1 (S_2 \ A) \qquad S \ \Gamma = \{x: S \ B | x:B \in \Gamma\} \qquad S \langle \Gamma ; A \rangle = \langle S \ \Gamma ; S \ A \rangle\]
	\begin{itemize}
		\item If there is a substitution $S$ such that $S \ A = B$ then $B$ is the \textit{substitution instance} of $A$.
		\item $Id_S$ (\textit{identity substitution}) maps every type variable to itself.
	\end{itemize}
\end{definitionbox}

For each typeable term $M$ there is a principal pair:
\[\langle \Pi ; P \rangle \text{ where } \Pi \textit{ is a context and } P \in \mathcal{T}_C \text{ such that } \forall \Gamma, A \in \mathcal{T}_C . \exists \text{ substitution }S . [S \langle \Pi ; P \rangle = \langle \Gamma; A \rangle]\]

\begin{tcbraster}[raster columns=2,raster equal height]
	\begin{definitionbox}{Soundness}
		A logical system is sound if every formula provable using the system is logically valid according to the semantics of the system.
		\[Provable \Rightarrow True\]
	\end{definitionbox}
	\begin{definitionbox}{Completeness}
		A logical system is complete if any true statement can be proved using the system.
		\[True \Rightarrow Provable\]
	\end{definitionbox}
\end{tcbraster}

This definition is sound, for every substitution $S$:
\[\text{if there is a derivation for } \Gamma \vdash_C M : A \text{ then we can  construct a derivation for } S \ \Gamma \vdash_C M : S \ A\]


\subsection{Unification}

\begin{definitionbox}{Robinson's Unification}
	\[\begin{matrix*}[l]
			unify & \varphi & \varphi & = (\varphi \mapsto \varphi) \\
			unify & \varphi & B & = (\varphi \mapsto B)  \text{ given } \varphi \text{does not occur in B} \\
			unify & A & \varphi & = unify \ \varphi \ A \\
			unify &(A \to B) & (C \to D) & = S_1 \circ S_2 \text{ where} \\
			& & & \qquad S_1 = unify \ A \ C \\
			& & & \qquad S_2 = unify \ (S_1 \ B) \ (S_1 \ D)\\
		\end{matrix*}\]
	\begin{itemize}
		\item Unification is associative and commutative
		\item It returns the most general unifier of two types (the \textit{common substitution instance})
	\end{itemize}
\end{definitionbox}

\textit{Robinson's Unification} can be generalised to unify contexts.
\[\begin{matrix*}[l]
		unifyContexts & (\Gamma_1, x:A) & \Gamma_2 & = unifyContexts \ \Gamma_1 \ \Gamma_2 \text{ given }x \text{ does not occur in } \Gamma_2 \\
		unifyContexts & \emptyset & \Gamma_2 & = Id_S \\
		unifyContexts & (\Gamma_1, x:A) & (\Gamma_2, x:B) & = S_1 \circ S_2 \text{ where}\\
		& & & \qquad S_1 = unify \ A \ B \\
		& & & \qquad S_2 = unifyContexts \ (S_1 \ \Gamma_1) \ (S_1 \ \Gamma_2) \\
	\end{matrix*}\]

\subsection{Curry Principle Pair}

\begin{definitionbox}{Curry Principle Pair}
	Every term $M$ has a \textit{(Curry) Principle Pair} defined as $pp_c \ M = \langle \Pi ; P \rangle$ by:
	\[\begin{matrix*}[l]
			pp_c & x & = \langle x: \varphi ; \varphi \rangle \text{ where } \varphi \text{ is fresh} \\
			\\
			pp_c & \lambda x. M & = \begin{cases}
				\langle \Pi' ; A \to P \rangle     & (\Pi = \Pi', x:A) \\
				\langle \Pi; \varphi \to P \rangle & (x \not\in \Pi)   \\
			\end{cases} \\
			& & \qquad \begin{matrix*}[l]
				\text{ where } & \langle \Pi; P \rangle &= pp_c \ M \\
				& \varphi & \text{ is fresh}  \\
			\end{matrix*} \\
			\\
			pp_c & M \ N & = S_2 \circ S_1 \langle \Pi_1 \cup \Pi_2 ; \varphi \rangle \\
			& & \qquad \begin{matrix*}[l]
				\text{ where } & \langle \Pi_1 ; P_1 \rangle & = pp_c \ M \\
				& \langle \Pi_2 ; P_2 \rangle & = pp_c \ N \\
				& S_1 & = unify \ P_1 \ (P_2 \to \varphi) \\
				& S_2 & = unifyContexts \ (S_1 \ \Pi_1) \ (S_1 \ \Pi_2) \\
				& \varphi & \text{ is fresh} \\
			\end{matrix*} \\
		\end{matrix*}\]
\end{definitionbox}

Substitution is complete:
\[\forall \Gamma, M \in \Lambda, A \in \mathcal{T}_c . [\Gamma \vdash_c M : A \Rightarrow \exists \Pi, P \in \mathcal{T}_c, S . [pp_c \ M = \langle \Pi ; P \rangle \land s \ \Pi \subseteq \Gamma \land S \ P = A]]\]
