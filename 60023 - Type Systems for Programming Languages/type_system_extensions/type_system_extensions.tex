\chapter{Extensions to Type Systems}

\section{Data Structures}
\begin{tabular}{l l l}
	Tuples & Structs/Data Classes/Records & Combine data, equivalent to product. \\
	Choice & Enums/Variants/Tagged Unions & Choose between variants of data.     \\
\end{tabular}
\subsection{Pairing}
We extend the grammar of types to:
\[A, B ::= \dots \ | \ A \times B | A + B\]
We can then extend the $\lambda$-calculus to:
\[E ::= \dots \ | \ \langle E_1 E_2 \rangle \ | \ left (E) \ | \ right(E) \]
And can add the following rules to the curry's type assignment system:
\[\ninfer{$Pair$}{\Gamma \vdash E_1 : A \quad \Gamma \vdash E_2 : B}{\Gamma \vdash \langle E-1, E_2 \rangle : A \times B}
	\qquad \ninfer{$left$}{\Gamma \vdash E : A \times B}{\Gamma \vdash left (E) : A}
	\qquad \ninfer{$right$}{\Gamma \vdash E : A \times B}{\Gamma \vdash right (E) : B}\]
The reduction rules for $left$ and $right$ are as follows:
\[\begin{split}
		left \ \langle E_1, E_2 \rangle & \to E_1 \\
		right \ \langle E_1, E_2 \rangle & \to E_2 \\
	\end{split} \qquad
	E \to E' \Rightarrow \begin{cases}
		\langle E', E_2 \rangle & \to \langle E', E_2 \rangle \\
		\langle E_1, E \rangle  & \to \langle E_1, E' \rangle \\
		left (E)                & \to left (E')               \\
		right (E)               & \to right (E')              \\
	\end{cases}\]
Here $left$ and $right$ are constructors (in the same spirit as seem with pattern matching), we could add a rule to reconstruct tuples as below, but this would remove confluence (this is the surjective pairing mentioned in pattern matching).
\[\langle left(E), right(E) \rangle \to E\]

\subsection{Disjoint Unions}
We can then extend the $\lambda$-calculus to:
\[E ::= \dots \ | \ case(E_1, E_2, E_3) \ | \ inj \cdot l (E) \ | \ inj \cdot r (E)\]
Here the unions can only be of two types, and $case$(expression, if A, if B) is a match/case of statement. $inj$ is used to construct the left or right type.
And can add the following rules to the curry's type assignment system:
\[\ninfer{$case$}{\Gamma \vdash E_1 : A + B \quad \Gamma \vdash E_2 : A \to C \quad \Gamma \vdash E_2 : B \to C}{\Gamma \vdash case(E_1, E_2, E_3) : C}\]
\[\ninfer{$inj \cdot l$}{\Gamma \vdash E : A}{\Gamma \vdash inj \cdot l (E) : A + B} \qquad
	\ninfer{$inj \cdot r$}{\Gamma \vdash E : B}{\Gamma \vdash inj \cdot r (E) : A + B} \]
The reduction rules for the new constructors are as follows:
\[\begin{split}
		case(inj \cdot l (E_1), E_2, E_3) &\to E_2 \ E_1 \\
		case(inj \cdot r (E_1), E_2, E_3) &\to E_3 \ E_1 \\
	\end{split} \qquad E \to E' \Rightarrow \begin{cases}
		case(E,E_2,E_3)  & \to case(E',E_2,E_3)  \\
		case(E_1,E,E_33) & \to case(E_1,E',E_33) \\
		case(E_1,E_2,E)  & \to case(E_1,E_2,E')  \\
		inj \cdot l (E)  & \to inj \cdot l (E')  \\
		inj \cdot r (E)  & \to inj \cdot r (E')  \\
	\end{cases}\]

\section{Recursive Types}
Recursive types are required for many types of data structure (single linked lists, trees, other structures with unbounded size).
\begin{definitionbox}{Unit Type}
	A unit type is an empty type containing no data.
	\begin{itemize}
		\item Considered an empty tuple
		\item Supported by many languages (e.g Rust, Haskell)
	\end{itemize}
\end{definitionbox}

To properly express recursive types, many programming languages include a unit type (to be used as the base case for some recursive types).
\[\ninfer{unit}{}{\Gamma \vdash () : \text{unit}}\]
\begin{minipage}{.5\textwidth}
	We can extend the grammar of types to:
	\[A, B = \dots \ | \ X \ | \ \mu X . A\]
\end{minipage}
\begin{minipage}{.5\textwidth}
	$=_\mu$ is the smallest equivalence relation containing:
	\[\mu X.A =_\mu A[\mu X.A/X]\]
\end{minipage}

\begin{examplebox}{lists}
	Define a singly linked list using the recursive types scheme described in this section.
	\tcblower
	\[[B] \triangleq \mu X. unit + (B \times X)\]
	We can show this using $=_\mu$ as:
	\[\begin{matrix*}[l]
			[B] &\triangleq& \mu X . unit + (B \times X) \\
			&=_\mu& (unit + (B \times X))[\mu X . unit + (B \times X)/X] \\
			&=& (unit + (B \times \mu X . unit + (B \times X))) \\
			&=& (unit + (B \times [B])) \\
		\end{matrix*}\]
\end{examplebox}

\subsection{Equi-recursive Approach}
\[\ninfers{$\mu$}{\Gamma \vdash E : A}{\Gamma \vdash E : B}{A =_\mu B}\]
Hence we can now type the list as:
\[\ainfer{$\mu$}{
	\ainfer{$inj \cdot l$}{
		\ainfer{unit}{}{
			\Gamma \vdash () : \text{unit}
		}
	}{
		\Gamma \vdash inj \cdot l () : \text{unit} + (B \times [B])
	}
	}{
	\Gamma \vdash inj \cdot l () : [B]
	}\]

\[\ainfer{$\mu$}{
	\ainfer{$inj \cdot r$}{
	\ainfer{Pair}{
	\infer{\dots}{\Gamma \vdash E_1 : B} \quad \infer{\dots}{\Gamma \vdash E_2 : [B]}
	}{
	\Gamma \vdash \langle E_1, E_2 \rangle : B \times [B]
	}
	}{
	\Gamma \vdash inj \cdot r \langle E_1 , E_2 \rangle : \text{unit} + (B \times [B])
	}
	}{
	\Gamma \vdash inj \cdot r \langle E_1, E_2 \rangle : [B]
	}\]
\begin{prosbox}
	This approach requires no explicit type annotations or declarations (full type inference is preserved)
\end{prosbox}
\begin{consbox}
	Meaningless program have types, for example self application can be typed.
	\[\lambda x . x \ x : \mu X.X \to \varphi\]
\end{consbox}

\subsection{Iso-recursive Approach}
Here syntactic markers for fold and unfold are added, recursive types are folded and unfolded on demand.
\begin{prosbox}
	Disallows typing of self-application, as is the case with \textit{equi-recursive types}.
\end{prosbox}

The syntax is extended with:
\[E ::= \dots \ | \ fold(E) \ | \ unfold(E) \]
The following reduction rule is added:
\[unfold(fold(E)) \to E \quad E \to E' \Rightarrow \begin{cases}
		unfold(E) & \to unfold(E') \\
		fold(E)   & \to fold(E')   \\
	\end{cases}\]
We then must add the type assignment rules:
\[\ninfer{fold}{
		\Gamma \vdash E : A[\mu X . A / X]
	}{
		\Gamma \vdash fold(E) : \mu X.A
	} \qquad \ninfer{unfold}{
		\Gamma \vdash E : \mu X . A
	}{
		\Gamma \vdash unfold(E) : A[\mu X.A / X]
	}\]
\[\ainfer{fold}{
	\ainfer{$inj \cdot l$}{
		\ainfer{unit}{}{
			\Gamma \vdash () : \text{unit}
		}
	}{
		\Gamma \vdash inj \cdot l () : \text{unit} + (B \times [B])
	}
	}{
	\Gamma \vdash fold(inj \cdot l ()) : [B]
	}\]

\[\ainfer{fold}{
	\ainfer{$inj \cdot r$}{
	\ainfer{Pair}{
	\infer{\dots}{\Gamma \vdash E_1 : B} \quad \infer{\dots}{\Gamma \vdash E_2 : [B]}
	}{
	\Gamma \vdash \langle E_1, E_2 \rangle : B \times [B]
	}
	}{\Gamma \vdash inj \cdot r \langle E_1 , E_2 \rangle : \text{unit} + (B \times [B])}
	}{
	\Gamma \vdash fold(inj \cdot r \langle E_1, E_2 \rangle) : [B]
	}\]


\section{Recursive Data Types}
Rather than specifically use $\mu . X . A$ for fold/unfold, we can instead generalise to fold/unfold any type. This allows us to build recursive data types.
\[\ninfer{$fold_{\mu X . A}$}{
		\Gamma \vdash E : A[\mu X . A / X]}{\Gamma \vdash fold_{\mu X . A}(E) : \mu X . A
	} \qquad \ninfer{$unfold_{\mu X . A}$}{
		\Gamma \vdash E : \mu X . A
	}{
		\Gamma \vdash unfold_{\mu X . A} (E) : A[\mu X . A / X]
	}\]

We can hence create some identifier for a recursive type, and then use the relevant fold and unfolds for it.
\[A = \text{unit} + (B \times A) \qquad \text{with identifier } [A]\]

More generally a recursive data type is defined as $C \overset{\rightharpoonup}{\varphi} = A_C[\overset{\rightharpoonup}{\varphi}]$.
\\
\\ The syntax can be extended by:
\[E ::= \dots \ | \ fold_C(E) \ | \ unfold_C(E)\]
We also add the reduction rules as:
\[unfold_C(fold_C(E)) \to E \qquad E \to E' \Rightarrow \begin{cases}
		unfold_C(E) & \to unfold_C(E') \\
		fold_C(E)   & \to fold_C(E')   \\
	\end{cases}\]
With type assignment rules:
\[\ninfer{$fold_C$}{
	\Gamma \vdash E : A_C[\overset{\rightharpoonup}{B}]
	}{
	\Gamma \vdash fold_C(E) : C \overset{\rightharpoonup}{B}
	} \qquad \ninfer{$unfold_C$}{
	\Gamma \vdash E : C \overset{\rightharpoonup}{B}
	}{
	\Gamma \vdash unfold_C(E) : A_C[\overset{\rightharpoonup}{B}]
	}\]
for every type definition $C \overset{\rightharpoonup}{\varphi} = A_C[\overset{\rightharpoonup}{\varphi}]$.

