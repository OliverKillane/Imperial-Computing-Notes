\chapter{Intersection Types}
We can extend the curry types with a new constructor $\cap$, and limit its use to the left-hand side of arrows.
\\
\\ To allow this we use a two-level grammar to define $\mathcal{T}_s$ (strict types)
\[\begin{matrix*}[l]
		A & ::= \varphi \ | \ (\sigma \to A) \\
		\sigma & ::= (A_1 \cap \dots \cap A_n) \quad (n \geq 0) \\
	\end{matrix*}\]

On $\mathcal{T}_s$ we define the relation $\leq$ as the smallest relation such that:
\[\begin{matrix*}[l]
		\forall 1 \leq i \leq n . \ [A_1 \cap \dots \cap A_n \leq A_i] & (n \geq 1)  & \text{\textit{The intersection is smaller than any contained type.}}\\
		\forall 1 \leq i \leq n . \ [\sigma \leq A_i] \Rightarrow \sigma \leq A_1 \cap \dots \cap A_n & (n \geq 0) & \text{\textit{If smaller than all types, its smaller than intersection.}}\\
		\sigma \leq \tau \leq \rho \Rightarrow \sigma \leq \rho & & \text{\textit{It is transitive.}}\\
	\end{matrix*}\]

We can define a relation $\sim$ on types (considered \textit{equals}) as:
\[\begin{split}
		\sigma \leq \tau \leq \sigma &\Rightarrow \sigma \sim \tau \\
		\sigma \sim \tau \land A \sim B & \Rightarrow \sigma \to A \sim \tau \to B \\
	\end{split}\]
Conventions:
\begin{itemize}
	\item We will consider types modulo $\sim$.
	\item $\bigcap_nA_i \triangleq A_1 \cap \dots A_n$
	\item $\bigcap_0A_i = \top$ where $\top$ is used for subterms that will disappear during reduction (and hence we do not care about)
\end{itemize}

\[\textit{A statement is: } \underset{\textit{subject}}{M} : \underset{\textit{predicate}}{\sigma}\]
A context $\Gamma$ is a set of statements with distinct variables as subjects (same as previously defined)
\\
\\ The relations for $\leq$ and $\sim$ can be extended to context by:
\[\begin{matrix*}[l]
		\Gamma \leq \Gamma' & \Leftrightarrow \forall x : \tau \in \Gamma' \exists x : \sigma \in \Gamma . \ [\sigma \leq \tau] \\
		\Gamma \sim \Gamma' & \Leftrightarrow \Gamma \leq \Gamma' \leq \Gamma \\
	\end{matrix*}\]

We can find the intersection of contexts as:
\[\begin{matrix*}[l]
		\Gamma_1 \cap \Gamma_2 \triangleq & \{x: \sigma \cap \tau &| x:\sigma \in \Gamma_1 & \land x:\tau \in \Gamma_2 &\} & \cup \\
		& \{x : \sigma &| x : \sigma \in \Gamma_1 & \land x \not\in \Gamma_2 &\} & \cup \\
		& \{x : \tau &| x : \tau \in \Gamma_2 & \land x \not\in \Gamma_1 &\} & \\
	\end{matrix*}\]
Conventions:
\begin{itemize}
	\item $\bigcap_{\underline{n}}\Gamma_i \triangleq \Gamma_1 \cap \dots \cap \Gamma_n$
	\item $\Gamma \cap \{x:\sigma\} = \Gamma \cap x:\sigma$
\end{itemize}

\section{Type Assignment}
\textit{Strict type assignment} and \textit{strict derivations} are defined through an inference system:
\\ \begin{minipage}{.5\textwidth}
	\[\ninfers{$Ax$}{}{\Gamma, x: \bigcap_nA_i \vdash_\cap x : A_i}{n \geq 1}\]
\end{minipage}
\begin{minipage}{.5\textwidth}
	\[\ninfers{$\cap I$}{\Gamma \vdash_\cap M:A_1 \quad \dots \quad \Gamma \vdash_\cap M : A_n }{\Gamma \vdash_\cap M : \bigcap_nA_i}{n \geq 0}\]
\end{minipage}
\\ \vspace{2mm}
\\ \begin{minipage}{.5\textwidth}
	\[\ninfer{$\to I$}{
			\Gamma, x: \sigma \vdash_\cap M : B
		}{
			\Gamma \vdash_\cap \lambda x . M : \sigma \to B
		}\]
\end{minipage}
\begin{minipage}{.5\textwidth}
	\[\ninfer{$\to E$}{
			\Gamma \vdash_\cap M : \sigma \to B \quad \Gamma \vdash_\cap N : \sigma
		}{
			\Gamma \vdash_\cap M \ N : B
		}\]
\end{minipage}
\begin{itemize}
	\item The $\cap I$ rule allows for $\bot$
\end{itemize}

\subsubsection{Generation Lemma}
\[\Gamma \vdash_\cap MN : A \Leftrightarrow \exists \sigma \in \mathcal{T}_s . \ [\Gamma \vdash_\cap M : \sigma \to A \land \Gamma \vdash_\cap N : \sigma]\]
\[\Gamma \vdash_\cap \lambda x . M : A \Leftrightarrow \exists \sigma, B . \ [A = \sigma \to B \land \Gamma, x: \sigma \vdash_\cap M : B]\]


\subsubsection{Weakening}
\[\Gamma \vdash_\cap M : \sigma \land \Gamma \subseteq \Gamma' \Rightarrow \Gamma' \vdash_\cap M : \sigma\]
Adding more variables to the environment (that are not free in the expression)

\subsubsection{Strengthening}
\[\Gamma \vdash_\cap M : \sigma \Rightarrow \{x:\tau \ | \ x : \tau \in \Gamma \land x \in fv(M)\} \vdash_\cap M : \sigma\]

As we can discard some types as $\top$ (\textit{"don't care"}) we can type some results where a discarded argument may not be typeable, but is not part of the end result, so is ignored.
\[\ainfer{$\to E$}{
		\ainfer{$\to E$}{
			\ainfer{$Ax$}{}{\emptyset, y : \top, z : A \vdash_\cap z : A}
		}{\emptyset, y : \top \vdash_\cap \lambda z . z : A \to A}
	}{
		\emptyset \vdash_\cap \lambda yz . z : \top \to A \to A
	}\]

\section{Subject Reduction and Normalisation}
\[\ainfer{$\to E$}{
		\ainfer{$\to I$}{
			\infer{
				\ainfer{$Ax$}{}{
					\Gamma, x: \bigcap_nB_i \vdash_\cap x : B_1
				} \quad \dots \quad \ainfer{$Ax$}{}{
					\Gamma, x: \bigcap_nB_i \vdash_\cap x : B_n
				}
			}{\vdots}
		}{\Gamma \vdash_\cap \lambda x . M : (\bigcap_nB_i) \to A}
		\quad \ainfer{$\cap I$}{
			\infer{\dots}{\Gamma \vdash_\cap N : B_1} \quad \dots \quad \infer{\dots}{\Gamma \vdash_\cap N : B_n}
		}{
			\Gamma \vdash_\cap \bigcap_nB_i
		}
	}{
		\Gamma \vdash_\cap (\lambda x. M) \ N : A
	}\]
In this system a variable can have more than one type, combined at an intersection using ($\cap I$).
\\
\\ We can use an empty ($\cap I$) to get a $\top$:
\[\ainfer{$\cap I$}{}{
		\Gamma \vdash M : \top
	}\]
\begin{itemize}
	\item Types are no invariant by $\eta$ reduction.
	\item
\end{itemize}

\section{Rank 2 and ML}
It is possible to limit depth at which the intersection type constructor can be used.
\begin{itemize}
	\item Rank $n$ being usage up to a depth of $n$ (rank $1$ only allows intersection at the top).
	\item Rank $2$ is enough to model ML's let constructor.
\end{itemize}
Recall that in ML we require a let construct in type assignment:
\[\ninfer{let}{\Gamma \vdash E_1 : \tau \quad \Gamma, x: \tau \vdash E_2 : B}{\mllet{x = E_1}{E_2 : B}}\]
This is used for when we want a term of the form $(\lambda x . E_2) \ E_1$m but cannot type as:
\[\infer{
	\infer{
		\infer{\vdots}{\Gamma \vdash E_1 : A[B/\varphi]} \quad \infer{\vdots}{\Gamma \vdash E_1 : A[C/\varphi]}
	}{
		\vdots
	}
	}{
	\Gamma \vdash E_2[E_2 / x] : D
	}\]
For example if some term with a polymorphic type is used with two different types within an expression, we need to type both separately.
\\
\\ By using Rank $2$ types we can type and intersect as we can associate two types with $E_1$:
\\ Where $\Gamma' = \Gamma, x:A[B/\varphi] \cap A[C/\varphi]$
\[\ainfer{let}{
		\ainfer{$\to I$}{
			\infer{
				\infer{
					\ainfer{$Ax$}{}{
						\Gamma' \vdash x : A[B/\varphi]
					} \quad \ainfer{$Ax$}{}{
						\Gamma' \vdash x : A[C/\varphi]
					}
				}{\vdots}
			}{\Gamma; \vdash E_2 : D}
		}{
			\Gamma \vdash \lambda x.E_2 : A[B/\varphi] \cap A[C/\varphi] \to D
		} \quad \ainfer{$\cap I$}{
			\infer{\vdots}{
				\Gamma \vdash E_1 : A[B/\varphi]
			} \quad \infer{\vdots}{
				\Gamma \vdash E_1 : A[C/\varphi]
			}
		}{
			\Gamma \vdash E_1 : A[B/\varphi] \cap A[C/\varphi]
		}
	}{
		\Gamma \vdash (\lambda x.E_2) \ E_1 : D
	}\]
\section{Approximation Results}
Approximation Theorem:
\[\Gamma \vdash_\cap M : \sigma \Leftrightarrow \exists \mathbf{A} \in \mathcal{A}M . \ [\Gamma \vdash_\cap \mathbf{A} : \sigma]\]
\begin{itemize}
	\item $\top$ can only appear in in the type of terms that are $\bot$.
	\item The assigned type of $M$ hence predicts part of the shape of the normal form of $M$.
	\item Type assignment with intersection types is undecidable.
\end{itemize}

\[\Gamma \vdash_\cap M : \sigma \land M \sqsubseteq M' \Rightarrow \Gamma \vdash_\cap M' : \sigma\]

\section{Characterisation of Head/Strong Normalisation}
\begin{definitionbox}{Head-Normalisation}
	\[\exists \Gamma, A . \ [\Gamma \vdash_\cap M : A] \Leftrightarrow M \text{ has head-normal form}\]
	A $\lambda$-term is in head normal form if it is an abstraction with a body that is not \textit{reducible}
\end{definitionbox}
\[M \text{ is strongly normalisable } \Leftrightarrow \Gamma \vdash_\cap M : A \land \top \text{ is not used in } A\]

\section{Principle Intersection Pairs}
For approximant $A \in \mathcal{A}$ we can define the principle part of $A$
\[\begin{matrix*}[l]
		pp & \bot & = \langle \emptyset \top \rangle \\
		\\
		pp & x & = \langle x : \varphi , \varphi \rangle \text{ where $\varphi$ is fresh}\\
		\\
		pp & (\lambda x . A) & = \begin{cases}
			\langle \Pi', \sigma \to P \rangle & (\Pi',x:\sigma = \Pi) \\
			\langle \Pi, \top \to P \rangle    & (x \not\in \Pi)       \\
		\end{cases} \\
		& & \begin{matrix*}[l]
			\text{ where } & \langle \Pi, P \rangle & = pp \ A \\
		\end{matrix*} \\
		\\
		pp & (x A_1 \dots A_n) & = \langle \bigcap_n\Pi_i \cap \{x: P_1 \to \dots \to P_n \to \varphi\}, \varphi \rangle \\
		& & \begin{matrix*}[l]
			\text{ where } & \langle \Pi_i, P_i \rangle & = pp \ A_i \\
			& \varphi & \text{ is fresh} \\
		\end{matrix*} \\
	\end{matrix*}\]
\subsubsection{Properties of $pp$}
\[pp(A) = \langle \Pi, P \rangle \Rightarrow \Pi \vdash_\cap A : P\]
\[\forall A \in \mathcal{A} . \ [A \neq \bot \Rightarrow \exists \Gamma, B . \ [\Gamma \vdash_\cap A : B]]\]

\subsubsection{Pairs for Arbitrary $\Lambda$-Terms}
We can define P as the set of all principle types of all approximants.
\[\mathcal{P} = \{ \langle \Pi, P \rangle \ | \ \exists A \in \mathcal{A} (pp(A) = \langle \Pi, P \rangle)\}\]
Hence we can now define $\mathcal{P}(M)$ as the set of all principle pairs for all approximants of $M$:
\[\mathcal{P}(M) = \{pp(A) \ | \ A \in \mathcal{A}(M)\}\]
\begin{itemize}
	\item If $\mathcal{P}(M)$ is finite then there is a pair $\langle \Pi, P \rangle = \sqcup \mathcal{P}(M)$ such that $\langle \Pi, P \rangle \in \mathcal{P}$ (the principle pair of $M$).
	\item If $\mathcal{P}(M)$ is infinite then the principle pair of $M$ is the infinite set of pairs $\mathcal{P}(M)$.
\end{itemize}



