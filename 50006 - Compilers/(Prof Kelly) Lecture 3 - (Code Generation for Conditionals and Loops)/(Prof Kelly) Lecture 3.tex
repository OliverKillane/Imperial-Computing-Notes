\documentclass{report}
    \title{50006 - Compilers - (Prof Kelly) Lecture 3}
    \author{Oliver Killane}
    \date{10/01/22}
\input{../50006 common.tex}
\begin{document}
\maketitle
\lectlink{https://imperial.cloud.panopto.eu/Panopto/Pages/Viewer.aspx?id=9e39f043-07bf-41bc-97ea-ae1401000fb5}

\section*{Simple Programming Language}
The grammar is expressed as:
\begin{center}
	\begin{tabular}{r l}
		$stat \to$  & $ident$ ':=' $exp$ $|$ $stat$ ';' $stat$ $|$ 'for' $ident$ 'from' $exp$ 'to' $exp$ 'do' $stat$ 'od \\
		$exp \to$   & $exp$ $binop$ $exp$ $|$ $unop$ $exp$ $|$ $ident$ $|$ $num$                                         \\
		$binop \to$ & '+' $|$ '-' $|$ '*' $|$ '/' $|$                                                                    \\
		$unop \to$  & '-'                                                                                                \\
	\end{tabular}
\end{center}

And abstract syntax tree as:
\codelines{Haskell}{1}{21}{stack compiler.hs}

\section*{Target Stack Machine}
\centerimage{width=0.8\textwidth}{stack machine.png}
Assembly instructions (Some are directives/pesudoinstructions):
\codelines{Haskell}{23}{34}{stack compiler.hs}

Pesudocode for execution behaviour:
\codelist{Python}{microcode.pesudo}

\begin{minipage}[t]{0.4\textwidth}
	\centerline{Typical Assembly}
	\codelist{Python}{assembly.pesudo}
\end{minipage}
\hfill
\begin{minipage}[t]{0.4\textwidth}
	\centerline{Compiler Code Generated}
	\codelist{Haskell}{code gen rep.hs}
\end{minipage}
The define directive (and assembly label) are directives to make the linker/assembler convert jumps to the label into jumps to the memory address of the instruction immediately after the label.

\section*{Translation (Naive Implementation)}
\codelist{haskell}{stack compiler.hs}

\section*{Intermediate Representations}
\compitem {
	\bullpara{Abstract Syntax Tree}{ Usually the first intermediate representation.
		Can include statements, operations and expressions in a uniform way (simple data structure)
		\\ Useful for sophisticated instruction selections and register allocation.
	}
	\bullpara{Flattened Control Flow Graph}{ Represents assembler-level code
		\\ Order of opeerations defines control flow, useful for loop-invariant code motion.
	}
	\bullpara{Dependency Based Graphs}{ More complex, used by most modern compilers.
		\\ Used for optimisations, can create 'static single assignment' graphs to deal with dependencies on mutable data.
	}
}

\end{document}
