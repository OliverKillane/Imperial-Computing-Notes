\chapter{Mutexes}

\section{Spinlocks}
\subsection{Test-and-Set Spinlock}
\begin{definitionbox}{Test-and-Set Spinlock}
    A single bit is used to determine if the lock is acquired or not.
    \begin{itemize}
        \item Threads can use an atomic test-and-set operation to attempt to acquire the lock.
        \item Avoids making system calls (expensive)
        \item Wastes CPU time while spinning - hence useful only when the expected wait / critical section is short.
        \item Poor worst case behaviour.
    \end{itemize}
    Hybrid locks that spin before using an OS provided lock mechanism exist to overcome the worst-case behaviour.
\end{definitionbox}

A simple spinlock can be implemented in C++ as:
\begin{minted}{Cpp}
#include <atomic>

class SpinLock {
  public:
    SpinLock() : _lock_bit(false) {}

    void Lock() {
      while (_lock_bit.exchange(true));
    }

    void Unlock() {
      _lock_bit.store(false)
    }
  
  private:
    std::atomic<bool> _lock_bit;
}
\end{minted}
This simple lock has an issue with \textit{cache thrashing}. 
\begin{itemize}
  \item Each core on a multicore CPU has its own caches (L2 (\textit{often}), L1 cache)
  \item A \textit{cache coherency protocol} is used to ensure different cache lines in different caches on different cores are coherent / if valid contain the correct value (not an outdated one)
  \item Typically when a location is written to in some cache, invalidation messages are sent to the other caches.
\end{itemize}
As in every attempt to acquire the lock, the test and set instruction writes to the lock 

\subsection{Local Spinning}
To reduce the frequency of invalidations a ticket based spinlock can be used.
\begin{itemize}
  \item First check if the lock is held (read - does not invalidate)
  \item Then if the lock is not held, attempt to access with an RMW operation (exchange) which writes \& invalidates.
\end{itemize}
\begin{minted}{Cpp}
class SpinLockLocalSpinning : public SpinLock {
  public:
    SpinLock() : _lock_bit(false) {}

    void Lock() {
      while (_lock_bit.exchange(true)) { // <- threads can still have a 'TAS fight' here
        // could not get the lock, so keep reading until it is free
        while (_lock_bit.load());
        // the lock is now free, so we attempt to acquire it again
      }
    }
}
\end{minted}
Bus traffic is still an issue - we want to reduce the rate of accesses. This will also help to reduce the contention.

\subsection{Fixed Backoff}
\begin{definitionbox}{Active Backoff}
  On each iteration of local spinning, do nothing / wait for some fixed time period. This waiting is \textit{active} (e.g spin in a loop).
\end{definitionbox}
We can implement active backoff by adding a fixed, redundant \mintinline{Cpp}{for} loop.
\begin{minted}{Cpp}
class SpinLockLocalActiveBackoff : public SpinLock {
  public:
    void Lock() {
      while (_lock_bit.exchange(true)) { // <- threads can still have a 'TAS fight' here
        // could not get the lock, so keep reading until it is free
        do {
          // volatile prevents the loop from being eliminated
          // spin for N iterations, we could also use library provided busy-wait functions
          for (volatile size_t i = 0; i < N; i++);
        } while (_lock_bit.load());
        // the lock is now free, so we attempt to acquire it again
      }
    }
}
\end{minted}

\begin{definitionbox}{Passive Backoff}
  On each iteration of local spinning, do nothing / wait for some fixed time period using a processor instruction that informs the CPU to wait/do nothing.
  \begin{itemize}
    \item \mintinline{asm}{pause} on x86, accessed through the \mintinline{Cpp}{_mm_pause()} \href{https://www.intel.com/content/www/us/en/docs/intrinsics-guide/index.html#techs=SSE2}{SSE2 intrinsic}
    \item \mintinline{asm}{YIELD} on ARM
  \end{itemize}
  These instructions inform the CPU to do nothing for some period of time.
  \begin{itemize}
    \item Less energy consumption than actively spinning.
    \item Does not induce an OS context switch, simply a \textit{do nothing} instruction.
  \end{itemize}
\end{definitionbox}
\begin{minted}{Cpp}
#include <emmintrin.h>

class SpinLockLocalPassiveBackoff : public SpinLock {
  public:
    void Lock() {
      while (_lock_bit.exchange(true)) { 
        // could not get the lock, so keep reading until it is free
        do {
          // pause N times
          for (size_t i = 0; i < N; i++) {
            _mm_pause()
          }
        } while (_lock_bit.load());
        // the lock is now free, so we attempt to acquire it again
      }
    }
}
\end{minted}

Both active and passive backoff use a fixed backoff period.
\begin{itemize}
  \item For a short backoff while little time is \textit{wasted} (lock free but threads are backed-off), threads are more likely to TAS at similar times (more potential for contention on the lock acquire)
  \item For a long backoff the timings of TAS diverge for different threads, however more time can be spent backed-off when the lock is acquirable.
\end{itemize}

\subsection{Exponential Backoff}
\begin{definitionbox}{Exponential Backoff}
  We can increase the backoff time exponentially for every iteration of local spinning.
  \begin{itemize}
    \item Start from a small value
    \item Double at each local spin iteration, until some maximum value is hit.
  \end{itemize}
\end{definitionbox}
\begin{minted}{Cpp}
// So we can specify backoffs: SpinLockLocalExponentialBackoff<4,4096>()
#template<
  size_t MIN_BACKOFFS = 2,
  size_t MAX_BACKOFFS = 2048
>
class SpinLockLocalExponentialBackoff : public SpinLock {
  public:
    void Lock() {
      size_t backoffs = MIN_BACKOFFS;
      while (_lock_bit.exchange(true)) { 
        // could not get the lock, so keep reading until it is free
        do {
          for (size_t i = 0; i < backoffs; i++) {
            _mm_pause()
          }
          // double the backoffs up to a limit
          backoffs = std::min(backoffs * 2, MAX_BACKOFFS)
        } while (_lock_bit.load());
      }
    }
}
\end{minted}
\begin{itemize}
  \item Much like with fixed backoffs, there may be \textit{wasted} time between the lock being released and a thread acquiring it (thread is backed-off/doing nothing)
  \item Higher contention means more failed acquisitions and longer backoff periods
\end{itemize}

\section{Ticket Locks}
\subsubsection{Unfairness}
There is a tension between fairness and performance for spinlocks:
\begin{itemize}
  \item Some thread may be unable to acquire a lock continually, while others may be able to reacquire frequently
  \item Threads reacquiring is good for temporal locality of cache (i.e a thread traversing a data structure will get better cache performance if it does not wait often - waiting allows opportunities for other threads to evict cached parts of the data structure)
  \item The lack of a fairness guarantee means thread starvation can occur, this can result in longer waits (e.g thread A is waiting on data from thread B, but thread B is being starved of a lock it needs by thread C to M)
\end{itemize}
A ticket lock a lock implementation that provides fairness guarantees.

\begin{definitionbox}{Ticket Lock}
  A spinlock that uses an integer \textit{ticket} to determine which of some waiting threads can run.
  \begin{itemize}
    \item To acquire, threads request a ticket, then spin until the lock matches their ticket.
    \item To release a thread sets the lock to serve the next ticket
  \end{itemize}
\end{definitionbox}

\begin{minted}{Cpp}
#include <atomic>

class TicketLock {
  public:
    SpinLock() : _next_ticket_(0), _serving_ticket_(0) {}

    void Lock() {
      const auto ticket = _next_ticket_.fetch_add(1);
      while (_serving_ticket_.load() != ticket) {
        _mm_pause();
      }
    }

    void Unlock() {
      _serving_ticket_.store(_serving_ticket_.load() + 1);
    }
  
  private:
    std::atomic<size_t> _next_ticket_;
    std::atomic<size_t> _serving_ticket_;
}
\end{minted}

\section{Futexes}
\begin{minipage}{.49\textwidth}
  \subsubsection{Spinlock}
  \begin{itemize}
    \item Thready busy-waits until the lock is free.
    \item Uses atomic RMW operations
    \item No system calls required, operates entirely in user-space
  \end{itemize}
  \textit{Staying awake is expensive.}
  \begin{prosbox}
    Useful when contention is low and critical sections are short
  \end{prosbox}
  \begin{consbox}
    High contention results in lots of wasteful \textit{spinning} rather than useful work
  \end{consbox}
\end{minipage}
\hfill
\begin{minipage}{.49\textwidth}
  \subsubsection{Sleeping Lock}
  \begin{itemize}
    \item Use of a system call to ask the OS to block the thread until the lock is available.
    \item While the thread is blocked, other threads can be run.
  \end{itemize}
  \textit{Going to sleep is expensive.}
  \begin{prosbox}
    Useful for long critical sections.
  \end{prosbox}
  \begin{consbox}
    Overhead of system call is problematic when critical section is short - lock will be free imminently.
  \end{consbox}
\end{minipage}
\begin{sidenotebox}{Confusing Terminology}
  note that \href{https://man7.org/linux/man-pages/man2/futex.2.html}{\mintinline{C}{futex}} is a name for a linux system call.
  \begin{itemize}
    \item Exposes kernel functionality (so is not userspace).
    \item Works on userspace data, hence can be used to implement synchronisation primitives (not just mutexes) that operate mostly in userspace.
  \end{itemize}
\end{sidenotebox}

\begin{definitionbox}{Fast Userspace Mutex (Futex)}
  
\end{definitionbox}

\subsection{Futex Operations}
\begin{minted}{C}
/* Wait on a futex
 * if *p != v -> Returns immediately
 * else       -> Adds the calling thread to the wait queue associated with p.
 */
void futex_wait(int *p, int v);
\end{minted}
\begin{itemize}
  \item The threads can be from different processes
\end{itemize}


\begin{minted}{C}
/* Wake up a wake_count threads in the thread queue associated with p */
void futex_wake(int *p, int wake_count);
\end{minted}
\begin{itemize}
  \item Typically \mintinline{C}{wake_count} is $1$ (wake next/one) or $INT_MAX$ (wake all)
\end{itemize}

Here \mintinline{C}{int *p} can be a pointer to any \textit{int-sized} data.

\subsection{Futex Implementation}
\begin{minted}{Cpp}
class MutexSimple {
  public:
    MutexSimple() : state_(kFree) {}

    void Lock() {
      while (state.exchange(kLocked) == kLocked) {
        syscall( SYS_futex, &state_, FUTEX_WAIT, kLocked, ...);
      }
    }

    void Unlock() {
      state_.store(kFree);
      syscall(SYS_futex, &state_, FUTEX_WAKE, 1, ...);
    }
  private:
    const int kFree = 0;
    const int kLocked = 1;
    std::atomic<int> state_;
};
\end{minted}