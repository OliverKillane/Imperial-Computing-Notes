\chapter{Side Channels}
\lectlink{https://imperial.cloud.panopto.eu/Panopto/Pages/Viewer.aspx?id=0bb6927a-f2f3-453e-af3f-af2a01120300}{Chapter 5 - Part 1: Sidechannel Vulnerabilities}

\begin{definitionbox}{Side Channel Attack}
    An exploit that attacks the implementation of an algorithm by observing the state of system it runs within. For example:
    \begin{itemize}
        \item Detecting what is cached through memory access times
        \item Power consumption, electromagnetic leaks and other physical effects
    \end{itemize}
\end{definitionbox}

\section{Exfiltration}

\begin{definitionbox}{Prime and Probe}
    Detect eviction of the attacker's working set, that is caused by the victim.
    \begin{enumerate}
        \item Attacker primes the cache by filling some sets with its own lines.
        \item Victim executes, once finished the attacker probes (timing its memory accesses) to see which of its lines were evicted.
    \end{enumerate}
    If a line has been evicted, then the victim accessed an address mapping to the same set.
    \begin{itemize}
        \item Requires the attacker to be able to force the start of victim execution.
    \end{itemize}
\end{definitionbox}

\begin{definitionbox}{Evict and Time}
    Detecting cache usage by the victim, by monitoring its performance after altering the cache.
    \begin{enumerate}
        \item Attacker causes victim to execute, and to preload the cache with its working set.
        \item Attacker evicts a specific line from the cache.
        \item Victim executes, attacker monitors execution time.
    \end{enumerate}
    By repeating this process for many lines, the attacker can see where the time to execute is lower, and hence which lines the victim was using.
    \begin{itemize}
        \item Requires the attacker to be able to force the start of victim execution.
    \end{itemize}
\end{definitionbox}

\begin{definitionbox}{Flush and Reload}
    Given the victim runs in the same address space. FLush the cache, let the victim run and study which lines it accessed.
    \begin{enumerate}
        \item Flush the shared line of interest (using dedicated instructions, or through contention by caching dummy data)
        \item Allow the victim to execute, the victim will then load any data it uses.
        \item Attacker then reloads the evicted line through an access, and measures time taken.
    \end{enumerate}
    A fast reload indicates the victim used that line.
\end{definitionbox}

\section{Shared State}
On a modern CPU a large amount of state is shared between cores.
\begin{itemize}
    \item L2 and lower Cache
    \item core interconnects/buses
\end{itemize}
If two threads run on thew same core (as with SMT, or HyperThreading):
\begin{itemize}
    \item Instruction and data caches, as well as TLB
    \item Branch predictor
    \item Prefetcher
    \item Rename Registers
    \item Dispatch Ports
\end{itemize}

\section{Triggering Victim Execution}
\begin{center}
    \begin{tabular}{p{.15\textwidth} p{.8\textwidth}}
        \textbf{System Call} & If the operating system can invoke the victim, or the operating system itself is the victim. \\
        \textbf{Lock Release} & A victim may be waiting on a held lock (e.g a lock on a file in the filesystem) \\
        \textbf{Call it} & If the victim is contained in the same address space, call it as a function. \\
    \end{tabular}
\end{center}
The latter point (\textit{"call it"}) applies as the attacker's code may be running in the same process as the victim, under some runtime system (e.g the JVM, or a browser's javascript engine).
\begin{definitionbox}{Language Based Security}
    The runtime system bundled with and provided by the language enforces separation between threads.
    \begin{itemize}
        \item i.e prevents pointer arithmetic, checks array index bounds
    \end{itemize}
\end{definitionbox}

\section{Side Channels in Speculative Execution}
Speculative execution of instructions can impact the state of the cache.
\begin{itemize}
    \item Can include instructions in code, that will be speculatively executed and as a result impact cache.
    \item e.g Bypassing array bounds checks using the CPU's speculative execution of the invalid instruction to modify cache.
\end{itemize}
This vulnerability is commonly known as \textit{Meltdown} \& \textit{spectre} are is present in most modern dynamically scheduled processors.

\lectlink{https://imperial.cloud.panopto.eu/Panopto/Pages/Viewer.aspx?id=77ffc1e8-4796-4ffe-9965-af2a011237e6}{Chapter 5 - Part 2: Attacking Other Processes and the OS}

\section{Mitigation}
\begin{center}
    \begin{tabular}{p{.1\textwidth} p{.9\textwidth}}
        \textbf{ASLR} & Address Space Layout Randomisation is where the operating system distributes user pages across memory. It can also randomly distribute the kernel pages locations within a process' address space. \\
        \textbf{KPTI} & Kernel Address Space Isolation is where the virtual address mapping of kernel pages is changed every time the kernel is entered. This requires a reload of the TLB and a substantial performance penalty. \\
    \end{tabular}
\end{center}

\section{Spectre v2}
Even with these mitigations, it is still possible by training the branch predictor to mispredict a call within the victim's code.
\begin{enumerate}
    \item Train the branch predictor to predict jumps to your subroutine. (Need to consider design of the BTB)
    \item Make the syscall to the kernel.
    \item Kernel code starts running, hits a call that is mispredicted to the attacker's code.
    \item Attacker's code speculatively executes before being flushed, but effect on cache, size channels is still present.
\end{enumerate}
This is called \textit{Spectre Variant 2}
\\
\\ There are several possible mitigations:
\begin{center}
    \begin{tabular}{p{.3\textwidth} p{.7\textwidth}}
        \textbf{Block Microarchitecture and Cache Side Channels} & Very difficult to impossible. \\
        \textbf{Reduce Accuracy of Probing} & Need to add noise to timers, but this will now affect other applications that need precision timing. \\
        \textbf{Prevent Branch Predictor Poisoning} & Can add an instruction to prevent branch prediction, but there is a performance cost. \\
        \textbf{Block Branch Predictor Contention} & Keep separate perdictions for each thread protection domain (ring/protection level). \\
    \end{tabular}
\end{center}
Another solution is \textit{retpoline} - an indirect branch using a return instruction.
\begin{minted}{NASM}
RP0:  call RP2                  ; Push address of 
RP1:  int 3                     ; A breakpoint (will be speculatively executed, but is a mispredict)
RP2:  mov [rsp], <jump target>  ; overwrite the return address, now return will go to <jump target>
RP3:  ret                       ; jump to jump target
\end{minted}


