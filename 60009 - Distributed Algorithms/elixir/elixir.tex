\chapter{Elixir}

\section{learning Elixir}
\begin{itemize}
    \item \textbf{\href{https://elixir-lang.org/getting-started/introduction.html}{Introduction To Elixir \& Installation}}
    \item \textbf{\href{https://elixir-lang.org/docs.html}{Elixir Documentation}} and \textbf{\href{https://hexdocs.pm/elixir/Kernel.html}{Standard Library}}
    \item \textbf{\href{https://elixir-lang.org/learning.html}{Elixir Learning Resources}}
    \item \textbf{\href{https://devhints.io/elixir}{Devhints Exlixir Cheatsheet}}
    \item \textbf{\href{https://github.com/itsgreggreg/elixir_quick_reference}{Elixir Quick Reference}}
    \item \textbf{\href{https://learnxinyminutes.com/docs/elixir/}{Learn Elixir in Y Minutes}}
\end{itemize}

\begin{examplebox}{Two Sum}
    Write a program to provide the two indexes of numbers in a list that sum to a given target.
    (This is the famous leetcode problem \href{https://leetcode.com/problems/two-sum/}{two sum}).
    \tcblower
    \begin{minted}{elixir}
defmodule Solution do
  @spec two_sum(nums :: [integer], target :: integer) :: [integer]
  def two_sum(nums, target) do
    nums
    |> Enum.with_index()
    |> Enum.reduce_while(%{}, fn {num, idx}, acc ->
      case Map.get(acc, target - num) do
        nil ->
          {:cont, Map.put(acc, num, idx)}
        val ->
          {:halt, [idx, val]}
      end
    end)
  end
end
    \end{minted}
    We could also write this recursively with a helper function
    \begin{minted}{elixir}
defmodule Solution do
@spec two_sum(nums :: [integer], target :: integer) :: [integer]
  def two_sum(nums, target) do
      two_sum_aux(nums, target, %{}, 0)
  end

  defp two_sum_aux([next | rest], target, prevs, index) do
      val = Map.get(prevs, target - next)
      if val != nil do
          [val, index]
      else
          two_sum_aux(rest, target, Map.put(prevs, next, index), index + 1)
      end
  end
end
    \end{minted}
\end{examplebox}

\begin{examplebox}{Add two numbers}
  Given The following linked list structure, write a program taking two numbers (represented in reverse as linked lists), and produce a linked list of their sum.
  (This is leetcode problem \href{https://leetcode.com/problems/add-two-numbers/}{add two numbers})
  \begin{minted}{elixir}
# Definition for singly-linked list.
defmodule ListNode do
  @type t :: %__MODULE__{
          val: integer,
          next: ListNode.t() | nil
        }
  defstruct val: 0, next: nil
end
  \end{minted}
  \tcblower
  \begin{minted}{elixir}
defmodule Solution do
  @spec add_two_numbers(l1 :: ListNode.t | nil, l2 :: ListNode.t | nil) :: ListNode.t | nil
  def add_two_numbers(l1, l2) do
    x = get_list(l1) + get_list(l2)
    if x == 0 do
        %ListNode{val: 0, next: nil}
    else
        to_list(x)
    end
  end

  defp get_list(node) do
    case node do
        %ListNode{val: v, next: n} -> v + 10 * get_list(n)
        nil -> 0
    end
  end

  defp to_list(n) do
    case n do
        0 -> nil
        i -> %ListNode{val: rem(i,10), next: to_list(div(i,10))}
    end
  end
end
  \end{minted}
\end{examplebox}

\section{The Elixir System}
\begin{definitionbox}{Elixir}
  A concurrent (with actors) and functional programming language used for fault tolerant distributed systems.
  \begin{itemize}
    \item A modernized successor language to \href{https://www.erlang.org/}{Erlang}
    \item Runs using BEAM (Erlang's virtual machine) and hence compatible with erlang
    \item Has many additions over erlang (protocols, streams and metaprogramming)
  \end{itemize}
\end{definitionbox}

\begin{definitionbox}{Elixir Processs}
  A lightweight user level thread (\href{https://en.wikipedia.org/wiki/Green_thread}{green threads}) managed by the runtime.
  \begin{itemize}
    \item Everything is a process.
    \item Processes are strongly isolated, when two processes interact it does not matter which nodes, or even machines they run on.
    \item Processes share no resources (cannot share variables), they can only interact through message passing.
    \item Process creation and destruction is fast.
    \item Processes interact by message passing.
    \item Processes have unique names, if a name ios known it can be used to pass messages
    \item Error handling is non-local.
    \item Processes do what they are supposed to do or fail.
  \end{itemize}
\end{definitionbox}
\begin{definitionbox}{Elixir Node}
  All elixir processes run within a node, a node can manage many processes (creation, scheduling, and garbage collection).
  \begin{itemize}
    \item A node runs as an OS process, potentially with several OS threads scheduled across several cores.
    \item Multiple nodes can run on a single machine (or virtual machine such as a docker container).
    \item A node can efficiently manage thousands to millions of elixir processes.
  \end{itemize}
\end{definitionbox}

Communication between processes is implemented through shared memory on the same machine and TCP when over a network. However processes are not exposed to this - the same primitives are used for inter and intra node/machine communication.

\section{Message Passing}
The \mintinline{elixir}{send} and \mintinline{elixir}{receive} statements are used for message passing.
\begin{minted}{elixir}
  # send somedata (any type) to process p
  send p, somedata

  # Wait until a message that matches the pattern is added to the message queue 
  # (or a timeout occurs), then remove it (potentially skipping over messages 
  # that do not match)
  receive do
    somepattern -> dosomething(somepattern)
    # ... some other patterns
  end
\end{minted}
\begin{itemize}
  \item Each process has its own message queue.
  \item Messages received are appended to the message queue of the receiving process.
  \item The sender does not wait for the message to be appended, it continues immediately after sending.
\end{itemize}

We can implement a basic client-server system in this way. Here we are using a component-based approach 
(split the program into components, each asynchronously message pass), by convention each component is an 
elixir module, modules can be instantiated in many processes \& (by convention) have a public \mintinline{elixir}{start()} function.

\begin{minted}[bgcolor=threadGrey]{elixir}
defmodule Cluster do
  def start do 
    # Spawn two processes, with the function start
    # Server.ex and Client.ex are modules containing a public start function
    # (Assuming we have tarted a client_node and server_node)
    s = Node.spawn(:'server_node@172.19.0.2', Server, :start, []) 
    c = Node.spawn(:'client_node@172.19.0.1', Client, :start, [])

    # We send the PIDs of the processes to each other, we can pattern match on 
    # atoms for convenience in receiving
    send s, { :bind, c }
    send c, { :bind, s }
  end
end
\end{minted}
\noindent
\begin{minipage}{.5\textwidth}
  \begin{minted}[bgcolor=threadYellow]{elixir}
  defmodule Server do
    def start do
      receive do
        { :bind, c } -> next(c) 
      end
   end
  
    # next is defined as private, here 
    # recursion is used for iteration. 
    # To avoid a stack overflow tail 
    # recursion is required
    defp next(c) do
      receive do
        { :circle, radius } -> 
          send c, { :result, 3.14 * radius 
                                  * radius}
        { :square, side } -> 
          send c, { :result, side * side}
      end
      next(c)
    end
  end
  \end{minted}
\end{minipage}
\begin{minipage}{.5\textwidth}
  \begin{minted}[bgcolor=threadBlue]{elixir}
  defmodule Client do
    def start do 
      receive do 
        { :bind, s } -> next(s) 
      end
    end
  
    defp next(s) do
      send s, { :circle, 1.0 }
      receive do 
        { :result, area } -> 
          IO.puts "Area is #{area}"
      end
      Process.sleep(1000) 
      next(s)
    end
  end





  \end{minted}  
\end{minipage}

