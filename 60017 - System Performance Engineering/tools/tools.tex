\chapter{Tools}

\section{Motivation}
C++ is a very complex language, that combines:
\begin{itemize}
    \item High level abstractions (templates, constructors \& destructors (to RAII), macros, polymorphism from inheritance/virtuals)
    \item Fine grained, manual control over memory (can use malloc, new, other allocators, in heap, or even on the stack) and memory access (all the power of C's pointer arithmetic)
    \item Undefined behaviour, to allow compilers leverage to optimise (e.g signed integer overflow is undefined in the C++ standard, so that compilers can implement it based on the target architecture)  
\end{itemize}
These provided abstractions are powerful, but can be difficult to check correctness for.

\begin{sidenotebox}{What about Rust?}
    Rust was largely invented as an attempt to solve the correctness issues encountered by C++, without the performance tradeoff made by many other languages with more memory safety (e.g swift, go).
    \\
    \\ The language is split into safe and unsafe, with necessarily unsafe (but high performance) code being written in unsafe to provide safe abstractions. Safe rust has several advantages over C++:
    \begin{itemize}
        \item No undefined behaviour.
        \item Any program with data races cannot compile (ownership \& borrow checker)
        \item Complete memory safety (no pointers, borrow checker)
        \item Sanitary macros (very powerful metaprogramming abstraction - comparable with languages such as Scala)
        \item Traits and generics (much better error messages than with templates, C++ is attempting this with concepts)
    \end{itemize}
    For unsafe rust sanitizer-like tools are available using \href{https://github.com/rust-lang/miri}{MIRI}.
    \\
    \\ Rust's performance is \href{https://programming-language-benchmarks.vercel.app/cpp-vs-rust}{comparable with C++}.
\end{sidenotebox}

\section{Correctness}

\subsection{Patterns}
\begin{definitionbox}{Resource Aquisition is Initialisation (RAII)}
    \unfinished
\end{definitionbox}

% for locks, file handles and smart pointers
% include std::move (not on slides)

\subsection{Compiler Warnings}
Compilers already provide a large number of warnings.
\begin{itemize}
    \item \href{https://gcc.gnu.org/onlinedocs/gcc/Warning-Options.html}{GCC provided warnings}
    \item \href{https://clang.llvm.org/docs/DiagnosticsReference.html}{Clang provided warnings}
    \item Linters can also be used to catch code quality issues, that could potentially result in correctness issues in future (hard to understand code leads to misunderstandings).
\end{itemize}
\begin{minted}{bash}
# Use all warnings and error on warnings (e.g to fail CI)
g++ ... -Wall -Werror ...
\end{minted}
It is possible to enable and disable compiler warnings from the code.
\begin{minted}{cpp}
// Selecting levels for -Wformat
#pragma GCC diagnostic warning "-Wformat"
#pragma GCC diagnostic error "-Wformat"
#pragma GCC diagnostic ignored "-Wformat
\end{minted}
More can be found in \href{https://gcc.gnu.org/onlinedocs/gcc/Diagnostic-Pragmas.html}{GCC's diagnostic pragmas documentation}, they can also be specified as attributes (see \href{https://en.cppreference.com/w/cpp/language/attributes}{attribute specifiers}).

\subsection{Sanitisers}
Sanitisers instrument code at compile time to detect errors at runtime.
\begin{itemize}
    \item Typically take large performance reduction
    \item Known bugs are present in sanitisers (though these are typically very rare edge cases)
    \item GCC provides several that can be found \href{https://gcc.gnu.org/onlinedocs/gcc/Instrumentation-Options.html}{here}
\end{itemize}

\begin{tcbraster}[raster columns=2,raster equal height]
\begin{definitionbox}{Address Sanitizer (Asan)}
    Detects out of bounds, use after free and other memory access related bugs.
    \begin{minted}{bash}

g++ ... -fsanitize=address
    \end{minted}
\end{definitionbox}
\begin{definitionbox}{Undefined Behaviour Sanitizer (Ubsan)}
    Checks for a large number of undefined behaviours in C++
    \begin{minted}{bash}
g++ ... -fsanitize=undefined
    \end{minted}
\end{definitionbox}
\end{tcbraster}
\begin{definitionbox}{Thread Sanitizer}
    A data-race detector, uses a vector clock algorithm to detect data races at runtime.
    \begin{minted}{bash}
g++ ... -fsanitize=thread
    \end{minted}
\end{definitionbox}
\begin{examplebox}{Malloc weirdness}
    Does the following code segfault? Use Asan to determine if the following code has any memory related bugs.
    \inputminted{cpp}{tools/code/memory_example.cpp}
    \tcblower
    \begin{minted}{bash}
g++ memory_example.cpp -o memory_example
./memory_example

# result of code
here
    \end{minted}
    There is no segfault (this is due to how malloc chunks work as explained \href{https://sourceware.org/glibc/wiki/MallocInternals}{here}), however we are accessing outside the bounds of the memory allocated.
    \\
    \\ When running with asan we see the warning:
    \begin{minted}{bash}
g++ memory_example.cpp -o memory_example -fsanitize=address
./memory_example

# asan output
==1094==ERROR: AddressSanitizer: heap-buffer-overflow on address 0x60b0000000ec at ...
# The error originates in "main"
    \end{minted}
\end{examplebox}

\subsection{Valgrind}
Valgrind provides tools to detect memory issues and data races.
\subsubsection{Memcheck}
\begin{minted}{bash}
valgrind --tool=memcheck program # place args on end
\end{minted}
\begin{itemize}
    \item Should include debugging symbols when compiling
    \item Much slower than sanitizers, but can take any binary.
    \item Memcheck documented \href{https://valgrind.org/docs/manual/mc-manual.html}{here}
\end{itemize}

\begin{examplebox}{Memcheck this!}
    Use Valgrind Memcheck to find the incorrect index in the previously checked program.
    \tcblower
    \begin{minted}{bash}
# include debug symbols
g++ memory_example.cpp -g -o memory_example
valgrind --tool=memcheck ./memory_example

# valgrind output:
==2375== Invalid write of size 4
==2375==    at 0x1091EB: main (memory_example.cpp:6)
==2375==  Address 0x4db2c7c is 4 bytes before a block of size 100 alloc'd
==2375==    at 0x483B7F3: malloc (in /usr/lib/x86_64-linux-gnu/valgrind/ ...
==2375==    by 0x1091DE: main (memory_example.cpp:5)
    \end{minted}
\end{examplebox}

\subsubsection{Helgrind}
\begin{minted}{bash}
valgrind --tool=helgrind program # place args on end
\end{minted}
Data race detection using valgrind.

\section{Performance}
\subsection{VTune}
\unfinished

\subsection{Perf}
\unfinished

\subsection{Coz}
\unfinished
