\documentclass{report}
    \title{50001 - Algorithm Analysis and Design - Lecture 10}
    \author{Oliver Killane}
    \date{18/11/21}

\input{../50001 common.tex}

\begin{document}
\maketitle
\lectlink{https://imperial.cloud.panopto.eu/Panopto/Pages/Viewer.aspx?id=62e2dbd5-4fba-4758-b2a6-addb016ec088}

\section*{List lookup}
\codelist{Haskell}{list lookup.hs}
As you ca  see \fun{!!} costs $O(n)$ as it may traverse the entire list.
\\
\\ If we want this access to be faster, we can use trees:
\codelist{Haskell}{list tree.hs}
This costs $O(\log n)$ as each recursive call acts on half of the remaining list.
\\
\\ However we have difficulty with insertion:
\begin{center}
	\begin{tabular}{l l l}
		Insert Quickly & $n : t = node (Leaf n) t$               & Effectively becomes a linked list, $\log n$ search time ruined. \\
		Insert Slowly  & Rebalance tree (e.g \keyword{AVL} tree) & Complex and no longer $O(1)$ insert.                            \\
	\end{tabular}
\end{center}

\section*{Random Access Lists}
A list containing elements that are either nothing, or a perfect tree with size the same as $2^{index}$.
\centerimage{width=\textwidth}{RAList.png}
The empty tree can be represented by a $Tip$ value (from the notes), or using type $Maybe(Tree)$ (from the lecture) where $Tree \ a = Leaf \ x \ | \ Node \ n \ l \ r$.
\\
\\ When we add to a tree, we add to the first element of the \struct{RAList}, if the invariant is breached (no longer perfect tree of size $2^0$) it can be combined with the next list over (if empty, place, else combine and repeat).
\\
\\ This way while the worst case insert is $O(n)$, our amortized complexity is $O(1)$ much as with increment.
\codelist{Haskell}{RAList.hs}
\end{document}
