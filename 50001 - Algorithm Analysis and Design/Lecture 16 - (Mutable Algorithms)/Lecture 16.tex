\documentclass{report}
    \title{50001 - Algorithm Analysis and Design - Lecture 16}
    \author{Oliver Killane}
    \date{18/12/21}
\input{../50001 common.tex}
\begin{document}
\maketitle
\lectlink{https://imperial.cloud.panopto.eu/Panopto/Pages/Viewer.aspx?id=940579d0-4ab4-41d0-a3a9-adf8008866f0}

\section*{Mutable Algorithms}
We can use \struct{STRef} \var{s} \var{a} to hold a mutable reference to
\var{a} that can be created, read and modified.
\codelist{Haskell}{st.hs}

We can use this to create a mutable version of fibonacci.
\codelist{Haskell}{fib.hs}

\section*{Mutable Datastructures}
\subsection*{Array}
\codelist{Haskell}{array.hs}
Each operation is assumed to take constant time.
\\
\\ For example, an algorithm to find the smallest natural number not in a list.
\codelist{Haskell}{minfree.hs}

\subsection*{Hash}
\codelist{Haskell}{hash.hs}
A hash generates an integer from some data. Typically range restricted
(e.g hashmap can hold a finite number of entries), and the hash function
should be designed to reduce collisions (two distinct data having the same hash).
\\
\\ Below is an example of a bucket based hash map, using linked list buckets.
\centerimage{width=0.6\textwidth}{hash diagram.png}
\end{document}
