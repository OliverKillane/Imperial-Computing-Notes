\documentclass{report}
    \title{50001 - Algorithm Analysis and Design - Lecture 15}
    \author{Oliver Killane}
    \date{29/11/21}
\input{../50001 common.tex}
\begin{document}
\maketitle
\lectlink{https://imperial.cloud.panopto.eu/Panopto/Pages/Viewer.aspx?id=e7c5b433-e2eb-4762-9a09-adee01382a82}

\section*{Randomized Treaps}
By using a random value for priority when inserting values into the treap, we can ensure a high likelihood of balancing, without complex balancing being required.
\\
\\ We can use this to create a randomized quicksort.
\codelist{Haskell}{randomized treap.hs}

\section*{Randomized Binary Trees}
We can balance a binary tree without using a treap, by inserting at the root (and rotating the tree to ensure it is ordered) with a certain probability.
\codelist{Haskell}{Randomized Binary Tree.hs}
For every insert we have chance $\cfrac{1}{n+1}$ of inserting at the root of the tree. Then this occurs, the contents are rotated to ensure the tree's ordering is maintained.
\\
\\ This means that there is a very high probability of balance being maintained, however correct results are only returned when distinct elements are inserted at most once.

\end{document}
