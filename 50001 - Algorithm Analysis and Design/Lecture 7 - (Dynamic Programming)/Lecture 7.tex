\documentclass{report}
    \title{50001 - Algorithm Analysis and Design - Lecture 7}
    \author{Oliver Killane}
    \date{13/11/21}

\input{../50001 common.tex}

\begin{document}
\maketitle
\lectlink{https://imperial.cloud.panopto.eu/Panopto/Pages/Viewer.aspx?id=6348fa64-2119-481e-b96b-adcf00dfa196}

\section*{Dynamic Programming}
A technique to efficiently calculate solutions to certain recursive problems.
\begin{enumerate}
	\item Describe an inefficient recursive algorithm.
	\item Reduce inefficiency by storing intermediate shared results.
\end{enumerate}

\section*{Fibonacci Sequence}
\subsubsection*{Fully Recursive}
\codelist{Haskell}{recursive fib.hs}
\begin{tabular}{l l}
	$T_{fib}(0)$ & $= 1$                               \\
	$T_{fib}(1)$ & $= 1$                               \\
	$T_{fib}(n)$ & $= 1 + T_{fib}(n-2) + T_{fib}(n-1)$ \\
\end{tabular}
\\ The complexity of this algorithm is $T_{fib}(n) \in O(2^n)$.
\subsubsection*{Saving Intermediate Results}
We can use a helper function which takes the remaining number of additions, and the two previous values.
\codelist{Haskell}{more parameter fib.hs}
\begin{tabular}{l l}
	$T_{fib}(0)$ & $= 1$                \\
	$T_{fib}(1)$ & $= 1$                \\
	$T_{fib}(n)$ & $= 1 + T_{fib}(n-1)$ \\
\end{tabular}
\\The complexity of this algorithm is $T_{fib}(n) \in O(n)$.
\\
\\ This way every value is calculated only once for each call. However values are not saved between calls.
\subsubsection*{Memoisation}
\codelist{Haskell}{memoising fib.hs}
This creates a large list, we must only index on the list to get the value.
By using an array we can reduce the time taken to get to the $n$th element.
\subsubsection*{Array Based Memoisation}
\codelist{Haskell}{memoisation functions.hs}
Hence we can make our algorithm:
\codelist{Haskell}{array memoised fib.hs}
Here we can do constant time lookups for values in the table. If a value is not present, it is lazily evaluated using other elements in the table.
\\
\\ In this way we only calculate each fibonacci number once, and only when we need it. Further it is saved for any subsequent calls to fib.

\section*{Edit-Distance}
The \keyword{Edit-Distance} Problem is concered with calculating the \keyword{Levenshtein} distance between two strings.
\sidenote{Levenshtein Distance}{
	The number of insertions, deletions \& updates required to convert one string into another.
	\[toil \to_{+1} oil \to_{+1} il \to_{+1} ill\]
}

\codelist{Haskell}{dist recursive.hs}
This problem becomes of order $O(3^n)$ as it recurs 3 ways for each call.
\\
\\ We can reuse results for two substrings through memoisation, first we make a new recursive version that uses the index we are checking in each string:
\codelist{Haskell}{dist indexed.hs}
We can then use \fun{tabulate} to create a memoised version.
\codelist{Haskell}{dist memoised.hs}
As there are at most $m \times n$ entires in the table, and each are calculated at most once, and the lookup time is constant (using arrays), the complexity is $O(mn)$.
\end{document}
