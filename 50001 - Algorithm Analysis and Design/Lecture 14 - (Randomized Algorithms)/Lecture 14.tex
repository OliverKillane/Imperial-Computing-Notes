\documentclass{report}
    \title{50001 - Algorithm Analysis and Design - Lecture 14}
    \author{Oliver Killane}
    \date{28/11/21}
\input{../50001 common.tex}
\begin{document}
\maketitle
\lectlink{https://imperial.cloud.panopto.eu/Panopto/Pages/Viewer.aspx?id=a562661f-da77-4d28-85a6-ade80168e6e5}

\section*{Randomized Algorithms}
An algorithm that uses random values to produce a result.
\begin{center}
	\begin{tabular}{l l l}
		\textbf{Algorithm Type} & \textbf{Running time} & \textbf{Correct Result} \\
		Monte Carlo             & Predicatable          & Unpredictably           \\
		Las Vegas               & Unpredictable         & Predictably             \\
	\end{tabular}
\end{center}

\section*{Random Generation}

Functions are deterministic (always map same inputs to same outputs), this is known as \keyword{Leibniz's law} or the \keyword{Law of indiscernibles}:
\[x = y \Rightarrow f x = f y\]
We can exhibit pesudo random behaviour using an input that varies \begin{tabular}{l l}
	explicitly & (e.g Random numbers through seeds) \\
	implicitly & (e.g Microphone or camera noise)   \\
\end{tabular}

\subsection*{Inside IO Monad}
We can use basic random through the IO monad like this:
\codelist{Haskell}{simple random.hs}
However using the \struct{IO monad} is too specific, we may want to use random numbers in other contexts.

\subsection*{StdGen}
In haskell we can use \struct{Stdgen}.
\codelist{Haskell}{stdgen.hs}
By passing the newly generated \struct{StdGen} we can generate new values based on the original seed.
\centerimage{width=\textwidth}{random seeds.png}

\subsection*{With Random Monad}
Rather than passing \struct{StdGen} seeds through the program, we can use the \struct{MonadRandom} monad which internally uses this value.


\section*{Randomized $\pi$}
\centerimage{width=0.6\textwidth}{randomized pi.png}
(Monte Carlo Algorithm - known number of samples, known running time per sample) To estimate $\pi$, find the proportion of randomly selected spots that are within the circle.
\begin{center}
	\begin{tabular}{l l}
		Area of square        & $2 \times 2 = 4$       \\
		Area of circle        & $\pi \times 1^2 = \pi$ \\
		Probability in circle & $\cfrac{\pi}{4}$       \\
	\end{tabular}
\end{center}
Once we have the proportion, we can multiply by 4 to get an estimate of $\pi$.
\codelist{Haskell}{montePi.hs}

\section*{Treaps}
Simultaneously a \keyword{Tree} and a \keyword{Heap}. Stores values in order, while promoting higher priority nodes to the top of the tree.
\centerimage{width=0.8\textwidth}{treap.png}
\codelist{Haskell}{treap.hs}
\centerimage{width=0.8\textwidth}{lnode and rnode.png}
\end{document}
